% Define the subtitle of the page
\title{Point-based evaluations}

% Begin the content of the page
\subsection{Point-based evaluations}

Edward supports point-based evaluations. The simplest example is
evaluating a model for classification. After inferring the posterior,
we can predict the label for each observation in the data and compare
it to their true labels. Edward implements a variety of metrics, such
as classification error and mean absolute error, based on these
predictions.

\subsubsection{Implementation}

Edward implements point-based evaluations through the
\texttt{predict()} function in the probability model. This
predicts the label given samples from the posterior $p(z \mid x)$
\begin{lstlisting}[language=Python]
class BayesianLinearRegression:
    def __init__(self, lik_variance=0.01, prior_variance=0.01):
        self.lik_variance = lik_variance
        self.prior_variance = prior_variance
        self.num_vars = 11

    def log_prob(self, xs, zs):
        x, y = xs['x'], xs['y']
        log_prior = -self.prior_variance * tf.reduce_sum(zs*zs, 1)
        b = zs[:, 0]
        W = tf.transpose(zs[:, 1:])
        mus = tf.matmul(x, W) + b
        y = tf.expand_dims(y, 1)
        log_lik = -tf.reduce_sum(tf.pow(mus - y, 2), 0) / self.lik_variance
        return log_lik + log_prior

    def predict(self, xs, zs):
        """Returns a prediction for each data point, averaging over
        each set of latent variables z in zs; and also return the true
        value."""
        x_test = xs['x']
        b = zs[:, 0]
        W = tf.transpose(zs[:, 1:])
        y_pred = tf.reduce_mean(tf.matmul(x_test, W) + b, 1)
        return y_pred
\end{lstlisting}

The \texttt{ed.evaluate()} method then evaluates a model as
\begin{lstlisting}[language=Python]
ed.evaluate('categorical_accuracy', my_model, my_variational, my_data)
ed.evaluate('mean_absolute_error', my_model, my_variational, my_data)
\end{lstlisting}
Swapping \texttt{my_data} with held-out data makes it easy to implement
cross-validation and other model assessment techniques.

Point-based evaluation applies generally to any setting, including
unsupervised tasks. For example, we can evaluate the likelihood of
observing the data.
\begin{lstlisting}[language=Python]
ed.evaluate('log_likelihood', my_model, my_variational, my_data)
\end{lstlisting}

Point-based evaluations are formally known as scoring rules (Gneiting
and Raftery, 2007) in decision theory. Scoring rules are commonly
used for model comparison, model selection, and model averaging.

\subsubsection{References}\label{references}

\begin{itemize}
\item
  Gneiting, T., & Raftery, A. E. (2007). Strictly Proper Scoring
  Rules, Prediction, and Estimation. Journal of the American
  Statistical Association, 102(477), 359–378.
\end{itemize}
