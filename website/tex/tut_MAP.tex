% Define the subtitle of the page
\title{Maximum a posteriori estimation}

% Begin the content of the page
\subsection{Maximum a posteriori estimation}

Maximum a posteriori (MAP) estimation is a form of approximate
posterior inference. It uses the mode as a point estimate of the
posterior distribution,
\begin{align*}
  z_\text{MAP}
  &=
  \arg \max_z
  p(z \mid x)\\
  &=
  \arg \max_z
  \log p(z \mid x).
\end{align*}
In practice, we work with logarithms of densities to avoid numerical underflow
issues.

The MAP estimate is the most likely configuration of the hidden patterns $z$
under the model. However, we cannot directly solve this optimization problem
because the posterior is typically intractable. To circumvent this, we use Bayes' rule to
optimize over the joint density,
\begin{align*}
  z_\text{MAP}
  &=
  \arg \max_z
  \log p(z \mid x)\\
  &=
  \arg \max_z
  \log p(x, z).
\end{align*}
This is valid because
\begin{align*}
  \log p(z \mid x)
  &=
  \log p(x, z) - \log p(x)\\
  &=
  \log p(x, z) - \text{constant in terms of} z.
\end{align*}
MAP estimation includes the common scenario of maximum
likelihood estimation as a special case,
\begin{align*}
  z_\text{MAP}
  &=
  \arg \max_z
  p(x, z)\\
  &=
  \arg \max_z
  p(x\mid z),
\end{align*}
where the prior $p(z)$ is flat, placing uniform probability over all
values $z$ supports. Placing a nonuniform prior can be thought of as
regularizing the estimation, penalizing values away from maximizing
the likelihood, which can lead to overfitting. For example, a normal
prior or Laplace prior on $z$ corresponds to L2 penalization, also
known as ridge regression, and L1 penalization respectively, also
known as the LASSO.

Maximum likelihood is also known as minimizing the cross entropy, where
for a data set $x=\{x_n\}$,
\begin{align*}
  z_\text{MAP}
  &=
  \arg \max_z
  \log p(x\mid z)
  \\
  &=
  \arg \max_z
  \sum_{n=1}^N \log p(x_n\mid z)
  \\
  &=
  \arg \min_z
  -\frac{1}{N}\sum_{n=1}^N \log p(x\mid z).
\end{align*}
The last expression can be thought of as an approximation to the cross
entropy between the true data distribution and $p(x\mid z)$,
using a set of $N$ data points.

\subsubsection{Gradient descent}

To find the MAP estimate of the latent variables $z$, we simply
calculate the gradient of the log joint density
\begin{align*}
  \nabla_z
  \log p(x, z)
\end{align*}
and follow it to a (local) optima.
Edward uses automatic differentiation, specifically with TensorFlow's
computational graphs, making this gradient computation both simple and
efficient to distribute.

\subsubsection{Implementation}

In Edward, we view MAP estimation as a special case of
\href{tut_KLqp.html}{$KL(q\|p)$ minimization}, where the variational
distribution is a point mass (delta function) distribution. This makes
explicit the additional assumptions underlying MAP estimation from the
viewpoint of variational inference: namely, it approximates the
posterior using a point to summarize the full distribution.

\begin{lstlisting}[language=Python]
class MAP(VariationalInference):
    def __init__(self, model, data=None, params=None):
        with tf.variable_scope("variational"):
            if hasattr(model, 'n_vars'):
                variational = Variational()
                variational.add(PointMass(model.n_vars, params))
            else:
                variational = Variational()
                variational.add(PointMass(0))

        super(MAP, self).__init__(model, variational, data)

    def build_loss(self):
        x = self.data
        z = self.variational.sample()
        self.loss = tf.squeeze(self.model.log_prob(x, z))
        return -self.loss
\end{lstlisting}

The \texttt{MAP} class inherits from \texttt{VariationalInference} and
initializes a point mass as the variational distribution. It requires
defining only one method:
\texttt{build_loss()}, which implicitly specifies the gradient of the
log joint. TensorFlow's automatic differentiation will apply to
\texttt {z}, thus backpropagating (applying the chain rule of
differentiation) to compute the gradient of the objective.

There is a nuance here. TensorFlow's optimizers are configured to
\emph{minimize} an objective function. So the gradient is set to be
the negative of the log density's gradient.

\subsubsection{References}\label{references}

\begin{itemize}
\item
  Murphy, K. (2012). Machine Learning: A Probabilistic Perspective. MIT Press.
\end{itemize}
