% Define the subtitle of the page
\title{Maximum a posteriori estimation}

% Begin the content of the page
\subsection{Maximum a posteriori estimation}

(We recommend reading the 
\href{tut_inference.html}{inference of probability models}
tutorial first.)

Maximum a posteriori (MAP) estimation seeks the mode of the posterior
\begin{align*}
  z_\text{MAP}
  &=
  \arg \max_z
  p(z \mid x)\\
  &=
  \arg \max_z
  \log p(z \mid x).
\end{align*}
We prefer working with logarithms of densities to avoid numerical underflow
issues.

The MAP estimate is the most likely configuration of the hidden patterns $z$
under the model. However, we cannot directly solve this optimization problem,
because the posterior is unknown. To circumvent this, we use Bayes' rule to
optimize over the joint density,
\begin{align*}
  z_\text{MAP}
  &=
  \arg \max_z
  \log p(z \mid x)\\
  &=
  \arg \max_z
  \log p(x, z).  
\end{align*}
This is valid because
\begin{align*}
  \log p(z \mid x)
  &=
  \log p(x, z) - \log p(x)\\
  &=
  \log p(x, z) - \text{constant in terms of} z.
\end{align*}

\subsubsection{Gradient ascent}

To find the MAP estimate of the latent variables $z$, we simply compute the
gradient of the log joint density
\begin{align*}
  \nabla_z
  \log p(x, z)
\end{align*}
and follow it to a local maximum. Automatic differentiation makes it easy to
compute this gradient without any additional mathematical derivation.

\subsubsection{Implementation}

In Edward, we view MAP estimation as a special case of 
\href{tut_KLqp.html}{$KL(q\|p)$ minimization} where the approximating variational
distribution is a point mass (delta function) distribution.

\begin{lstlisting}[language=Python]
class MAP(VariationalInference):
    def __init__(self, model, data=None, params=None):
        with tf.variable_scope("variational"):
            if hasattr(model, 'num_vars'):
                variational = Variational()
                variational.add(PointMass(model.num_vars, params))
            else:
                variational = Variational()
                variational.add(PointMass(0))

        super(MAP, self).__init__(model, variational, data)

    def build_loss(self):
        x = self.data
        z = self.variational.sample()
        self.loss = tf.squeeze(self.model.log_prob(x, z))
        return -self.loss  
\end{lstlisting}

The \texttt{MAP} class automatically initializes a variational point mass
distribution. The \texttt{build_loss} method implicitly specifies the gradient
of the log joint. TensorFlow's automatic differentiation will apply to \texttt
{z}, thus applying the chain rule of differentiation to compute the gradient of
the ELBO.

There is a nuance here. This method actually returns the \textbf{negative} log
joint density because TensorFlow's stochastic optimization routines are all
configured to \emph {minimize} an objective function.

