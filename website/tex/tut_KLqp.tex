% Define the subtitle of the page
\title{KL(q||p) minimization}

% Begin the content of the page
\subsection{$\text{KL}(q\|p)$ minimization}

(We recommend reading the 
\href{tut_inference_probability_models.html}{inference of probability models}
tutorial first.)

Variational inference is a powerful approach to approximating the posterior. The
core idea involves two steps:
\begin{enumerate}
   \item propose a tractable approximation to the posterior $q(z\;;\;\lambda)$,
   and
   \item match it to the posterior by tuning the variational
   parameters $\lambda$.
 \end{enumerate} 
This strategy converts the problem of computing the posterior $p(z \mid x)$ into
an optimization problem of minimizing some divergence metric
\begin{align*}
  \lambda^* 
  &=
  \arg\min_\lambda \text{divergence}( 
  p(z \mid x)
  ,
  q(z\;;\;\lambda)
  ).
\end{align*}

One form of variational inference minimizes the Kullback-Leibler divergence 
\textbf{from} $q(z\;;\;\lambda)$ \textbf{to} $p(z \mid x)$,
\begin{align*}
  \lambda^* 
  &=
  \arg\min_\lambda \text{KL}( 
  q(z\;;\;\lambda)
  \;\|\;
  p(z \mid x)
  )\\
  &=\;
  \mathbb{E}_{q(z\;;\;\lambda)}
  \big[
  \log q(z\;;\;\lambda)
  -
  \log p(z \mid x)
  \big].
\end{align*}
The KL divergence is a non-symmetric, information theoretic metric of
similiarity between two probability distributions. 

\subsubsection{The Evidence Lower Bound}

The above optimization problem is intractable because it directly depends on the
posterior $p(z \mid x)$. To tackle this, consider the following relationship
\begin{align*}
  \log p(x)
  &=
  \text{KL}( 
  q(z\;;\;\lambda)
  \;\|\;
  p(z \mid x)
  )\\
  &\quad+\;
  \mathbb{E}_{q(z\;;\;\lambda)}
  \big[
  \log p(x, z)
  -
  \log q(z\;;\;\lambda)
  \big]
\end{align*}
where the left hand side is the logarithm of the evidence 
$p(x) = \int p(x,z) \text{d}z$. (Try deriving this using Bayes rule!)

Since the evidence does not depend on the variational parameters $\lambda$,
we can minimize $\text{KL}(q\|p)$ by instead maximizing
\begin{align*}
  \text{ELBO}(\lambda)
  &=\;
  \mathbb{E}_{q(z\;;\;\lambda)}
  \big[
  \log p(x, z)
  -
  \log q(z\;;\;\lambda)
  \big],
\end{align*}
the Evidence Lower BOund. Here, both $p(x,z)$ and $q(z\;;\;\lambda)$ are
tractable, by design. The optimization problem we seek to solve becomes
\begin{align*}
  \lambda^*
  &=
  \arg \max_\lambda \text{ELBO}(\lambda).
\end{align*}

What does maximizing the ELBO do? Expanding the ELBO reveals
\begin{align*}
  \text{ELBO}(\lambda) 
  &=\;
  \mathbb{E}_{q(z \;;\; \lambda)}[\log p(x \mid z)]  
  - \text{KL}(q(z \;;\; \lambda) \;\|\; p(z)).
\end{align*}
Maximizing the ELBO encourages $q$ to place its mass to explain the data through
the likelihood $p(x \mid z)$. But the KL term penalizes $q$ as it drifts away
from the prior $p(z)$. 


