\title{Tutorial: Latent Space Models}

\subsection{Tutorial: Latent Space Models}

This tutorial correponds to code found here [FIX LINK]
  \href{https://github.com/blei-lab/edward/blob/master/examples/bayesian_linear_regression.py}{Bayesian
  linear regression}

The purpose of this tutorial is to show how one can use edward to fit a latent space model to network data from a neuroscience application.

It is structured as follows.
We first review some basic ingredients of probailistic modeling in edward.
Next, we define the model and specify implementation details. 
Finally we fit the model with MAP estimation.

\subsection{Probabilistic Modeling in Edward}
Fitting a model in edward requires few ingredients: Defining the model, loading the data into an edward data object and defining an inference procedure and running it.

\begin{lstlisting}[language=Python]
import edward as ed
import tensorflow as tf

class Latent_Space_Model():
    def __init__(...):
        ...
        self.num_vars = ...
 
    def log_prob(self,xs,zs):
        log_prior = ...
        log_like = ...
        return log_like + log_prior

model = Latent_Space_Model(...)
\end{lstlisting}
Note that the model has two important properties. We need to specify the field \texttt{self.num\_vars} which holds the number of model variables. We also need to specify a method for calculating the log probability of the model given data \texttt{xs} and latent variables \texttt{zs}. The method is \texttt{self.log_prob(xs,zs)}.

Next we load the data:
\begin{lstlisting}[language=Python]
from ed.data import load_connectome
data = load_connectome()
\end{lstlisting}
A description of the data can be found below.

Finally, we can instantiate a MAP inference algorithm for the model and the data and run it.
\begin{lstlisting}[language=Python]
inference = ed.MAP(model, data)
inference.run(n_iter=500, n_print=100)
\end{lstlisting}

\subsection{Latent Space Models}

networks are cool and ubiqutous. social networks. neuroscience. 
cite hoff (Hoff, 2002)
learn a latent representation for each node

\subsection{Data: C.Elegans Connectome}

Data comes from \href{http://www-personal.umich.edu/~mejn/netdata/}{here}.
It is a weighted, directed network representing the neural network of the nematode C. Elegans compiled by Duncan Watts and Steven Strogatz from original experimental data by White et al.

\subsubsection{References}
\begin{thebibliography}{}
\bibitem[\protect\astroncite{Hoff et~al.}{2002}]{hoff2002latent}
Hoff, P.~D., Raftery, A.~E., and Handcock, M.~S. (2002).
\newblock Latent space approaches to social network analysis.
\newblock {\em Journal of the american Statistical association},
  97(460):1090--1098.

\bibitem[\protect\astroncite{Watts and Strogatz}{1998}]{watts1998collective}
Watts, D.~J. and Strogatz, S.~H. (1998).
\newblock Collective dynamics of ‘small-world’networks.
\newblock {\em nature}, 393(6684):440--442.

\bibitem[\protect\astroncite{White et~al.}{1986}]{white1986structure}
White, J.~G., Southgate, E., Thomson, J.~N., and Brenner, S. (1986).
\newblock The structure of the nervous system of the nematode caenorhabditis
  elegans.
\newblock {\em Philos Trans R Soc Lond B Biol Sci}, 314(1165):1--340.
\end{thebibliography}

