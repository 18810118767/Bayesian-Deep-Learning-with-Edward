% Define the subtitle of the page
\title{Bayesian neural network}

% Begin the content of the page
\subsection{Bayesian neural network}

\begin{itemize}
  \item \textbf{Probability Model:} Bayesian neural network regression
  \item \textbf{Variational Model:} Fully factorized normal 
  \item \textbf{Data:} Simulated data
  \item \textbf{Inference:} Mean-field variational inference
  \item \textbf{Criticism:} None
\end{itemize}

Full Python code in: 
\href{https://github.com/blei-lab/edward/blob/master/examples/bayesian_nn.py}
{bayesian_nn.py}


\subsubsection{Probability Model}
Define the likelihood of an observation $(y_n, x_n)$
\begin{align*}
  p(y_n \mid \mathbf{z} \;;\; x_n, \sigma^2)
  &=
  \mathcal{N}(y_n \;;\; \mu(x_n\;;\;\mathbf{z}), \sigma^2)
\end{align*} 
where $\mu$ is a neural network with weights (latent) variables 
$\mathbf{z}$. Here $x_n$ is the (known) covariate and $\sigma^2$ is the
(known) variance of the observation.

Define the prior on $\mathbf{z}$ to be a factorized normal
\begin{align*}
  p(\mathbf{z})
  &=
  \mathcal{N}(\mathbf{z} \;;\; \mathbf{0}, I)
\end{align*} 

The model in Edward is 
\begin{lstlisting}[language=Python]
class BayesianNN:
    """
    Bayesian neural network for regressing outputs y on inputs x.

    p((x,y), z) = Normal(y | NN(x; z), lik_variance) *
                  Normal(z | 0, prior_variance),

    where z are neural network weights, and with known lik_variance
    and prior_variance.

    Parameters
    ----------
    layer_sizes : list
        The size of each layer, ordered from input to output.
    nonlinearity : function, optional
        Non-linearity after each linear transformation in the neural
        network; aka activation function.
    lik_variance : float, optional
        Variance of the normal likelihood; aka noise parameter,
        homoscedastic variance, scale parameter.
    prior_variance : float, optional
        Variance of the normal prior on weights; aka L2
        regularization parameter, ridge penalty, scale parameter.
    """
    def __init__(self, layer_sizes, nonlinearity=tf.nn.tanh,
        lik_variance=0.01, prior_variance=1):
        self.layer_sizes = layer_sizes
        self.nonlinearity = nonlinearity
        self.lik_variance = lik_variance
        self.prior_variance = prior_variance

        self.num_layers = len(layer_sizes)
        self.weight_dims = zip(layer_sizes[:-1], layer_sizes[1:])
        self.num_vars = sum((m+1)*n for m, n in self.weight_dims)

    def unpack_weights(self, z):
        """Unpack weight matrices and biases from a flattened vector."""
        for m, n in self.weight_dims:
            yield tf.reshape(z[:m*n],        [m, n]), \
                  tf.reshape(z[m*n:(m*n+n)], [1, n])
            z = z[(m+1)*n:]

    def mapping(self, x, z):
        """
        mu = NN(x; z)

        Note this is one sample of z at a time.

        Parameters
        -------
        x : tf.tensor
            n_data x D

        z : tf.tensor
            num_vars

        Returns
        -------
        tf.tensor
            vector of length n_data
        """
        h = x
        for W, b in self.unpack_weights(z):
            # broadcasting to do (h*W) + b (e.g. 40x10 + 1x10)
            h = self.nonlinearity(tf.matmul(h, W) + b)

        h = tf.squeeze(h) # n_data x 1 to n_data
        return h

    def log_prob(self, xs, zs):
        """Returns a vector [log p(xs, zs[1,:]), ..., log p(xs, zs[S,:])]."""
        # Data must have labels in the first column and features in
        # subsequent columns.
        y = xs[:, 0]
        x = xs[:, 1:]
        log_prior = -self.prior_variance * tf.reduce_sum(zs*zs, 1)
        mus = tf.pack([self.mapping(x, z) for z in tf.unpack(zs)])
        # broadcasting to do mus - y (n_minibatch x n_data - n_data)
        log_lik = -tf.reduce_sum(tf.pow(mus - y, 2), 1) / self.lik_variance
        return log_lik + log_prior

model = BayesianNN(layer_sizes=[1, 10, 10, 1], nonlinearity=rbf)
\end{lstlisting}

\subsubsection{Variational Model}
Define the variational model to be a fully factorized normal
\begin{align*}
  q(\mathbf{z} \;;\; \lambda)
  &=
  \mathcal{N}(\mathbf{z} \;;\; \lambda_\mu, \lambda_{\sigma^2} I).
\end{align*}

The model in Edward is
\begin{lstlisting}[language=Python]
variational = Variational()
variational.add(Normal(model.num_vars))
\end{lstlisting}


\subsubsection{Data}

Consider a simulated dataset
\begin{lstlisting}[language=Python]
def build_toy_dataset(n_data=40, noise_std=0.1):
    ed.set_seed(0)
    D = 1
    x  = np.concatenate([np.linspace(0, 2, num=n_data/2),
                         np.linspace(6, 8, num=n_data/2)])
    y = np.cos(x) + norm.rvs(0, noise_std, size=n_data).reshape((n_data,))
    x = (x - 4.0) / 4.0
    x = x.reshape((n_data, D))
    y = y.reshape((n_data, 1))
    data = np.concatenate((y, x), axis=1) # n_data x (D+1)
    data = tf.constant(data, dtype=tf.float32)
    return ed.Data(data)

data = build_toy_dataset()
\end{lstlisting}


\subsubsection{Inference}

Inference is mean-field variational inference
\begin{lstlisting}[language=Python]
inference = ed.MFVI(model, variational, data)
inference.run(n_iter=1000, n_print=10)
\end{lstlisting}
