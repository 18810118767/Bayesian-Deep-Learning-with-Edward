% Define the subtitle of the page
\title{Inference of probability models}

% Begin the content of the page
\subsection{Inference of probability models}

This tutorial asks the question: what does it mean to do inference in
probability models? This sets the stage for understanding how to
develop and implement inference procedures in Edward.

\subsubsection{The posterior}

How can we use a model $p(x,z)$ to analyze some data $x$? In other words,
what hidden patterns $z$ explain the data? We seek to infer these
hidden patterns using the model.

Inference for probability models leverages Bayes' rule to define the
\emph{posterior}
\begin{align*}
  p(z \mid x)
  &=
  \frac{p(x,z)}{\int p(x,z) \text{d}z}.
\end{align*}
The posterior is the distribution of the latent variables $z$, conditioned on
some (observed) data $x$. It provides a probabilistic description of the hidden
patterns that explain the data. Wrapping up our example, say you measured your
weight $7$ times and the posterior of your weight turns out to be
\begin{align*}
  p(z \mid x) &= \mathcal{N}(z\;;\; 68.21, 2.19).
\end{align*}
This is a complete description of the model's estimate of your weight and its
uncertainty around it.


\subsubsection{Computing the posterior}

Now we know what the posterior represents. How do we compute it? This is the
central computational challenge in inference.

The posterior difficult to compute because of the integral in the denominator.
This is often a high-dimensional integral that lacks an analytic (closed-form)
solution. Thus, computing the posterior means \emph{approximating} the
posterior.

There are many methods to approximate the posterior. Edward provides a scaffold
for these methods. We describe them in detail in the other inference
\href{tutorials.html}{tutorials}.
