% Define the subtitle of the page
\title{Bayesian Inference}

% Begin the content of the page
\subsection{Bayesian Inference}

This tutorial introduces the class of models supported in Edward, namely
Bayesian probability models. It asks the question: what does it mean to do
inference in Bayesian models? This sets the stage for
understanding how to develop and implement inference procedures in Edward.

\subsubsection{Bayesian models}

A Bayesian model has two components. The \emph{likelihood}
\begin{align*}
  p(x \mid z)
\end{align*}
is a probability distribution that describes how the data $x$ depend on some
latent random variables $z$. The likelihood models a random data generating
process, where the data $x$ are observed random variables conditioned on a
particular hidden pattern described by $z$. For example, consider measuring your
weight on a cheap scale; the likelihood model would describe the imprecisions
of the scale that reflect into the measurements.

Bayesian models additional posit a \emph{prior}
\begin{align*}
  p(z).
\end{align*}
The prior is another probability distribution that captures our prior beliefs on
the latent random variables $z$. It describes the kind of hidden patterns we
expect to have produced the data $x$. Continuing our example, the prior would
capture your prior belief about your (latent) weight. A potential prior might be
a normal distribution centered at the average weight of adults on Earth (
which is $\approx62$ kilograms).

Combining the likelihood and the prior specifies a Bayesian model,
\begin{align*}
  p(x,z)
  &=
  p(x \mid z)
  p(z)
\end{align*}
as a joint density over observed and latent random variables. 

\subsubsection{The posterior}

How can we use a model $p(x,z)$ to analyze some data $x$? In other words,
what hidden patterns $z$ explain the data? We seek to infer these
hidden patterns using the model.

The Bayesian approach to inference leverages Bayes rule to define the 
\emph{posterior}
\begin{align*}
  p(z \mid x)
  &=
  \frac{p(x,z)}{\int p(x,z) \text{d}z}.
\end{align*}
The posterior is the distribution of the latent variables $z$, conditioned on
some (observed) data $x$. It provides a probabilistic description of the hidden
patterns that explain the data. Wrapping up our example, say you measured your
weight $7$ times and the posterior of your weight turns out to be
\begin{align*}
  p(z \mid x) &= \mathcal{N}(z\;;\; 68.21, 2.19).
\end{align*}
This is a complete description of the model's estimate of your weight and its
uncertainty around it.


\subsubsection{Computing the posterior}

Now we know what the posterior represents. How do we compute it? This is the
central computational challenge in Bayesian inference.

The posterior difficult to compute because of the integral in the denominator.
This is often a high-dimensional integral that lacks an analytic (closed-form)
solution. Thus, computing the posterior means \emph{approximating} the
posterior.

There are many methods to approximate the posterior. Edward provides a scaffold
for these methods. We describe them in detail in the other inference 
\href{tutorials.html}{tutorials}.
