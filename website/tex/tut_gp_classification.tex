% Define the subtitle of the page
\title{Gaussian process classification}

% Begin the content of the page
\subsection{Gaussian process classification}

\subsection{(ON HOLD)}

\begin{itemize}
  \item \textbf{Probability Model:} Mixture of Gaussians
  \item \textbf{Variational Model:} Dirichlet $\times$ factorized normal
  $\times$ factorized inverse gamma
  \item \textbf{Data:} Simulated data
  \item \textbf{Inference:} Mean-field variational inference
  \item \textbf{Criticism:} None
\end{itemize}

Full Python code in: 
\href{https://github.com/blei-lab/edward/blob/master/examples/gp_classification.py}
{gp_classification.py}


\subsubsection{Probability Model}
Define the likelihood of an observation $x_n$
to be
\begin{align*}
  p(x_{n} \mid \pi, \mu, \Sigma)
  &=
  \sum_{k=1}^K \pi_k \, \mathcal{N}(x_n ; \mu_k, \sigma_k).
\end{align*}
The latent variable $\pi$ mixes $K$ Gaussian distributions, each 
characterized by mean $\mu_k$ and variance $\sigma_k$.

Define the prior on $\pi$ to be a Dirichlet 
\begin{align*}
  p(\pi)
  &=
  \text{Dirichlet}(\pi \;;\; \alpha \mathbf{1}_{K \times 1}).
\end{align*} 

Define the prior on $\mu$ to be a factorized normal 
\begin{align*}
  p(\mu)
  &=
  \prod_{k=1}^{K} \mathcal{N}(\mu_k \;;\; 0, \sigma^2).
\end{align*} 

Define the prior on $\sigma$ to be a factorized inverse gamma
\begin{align*}
  p(\sigma)
  &=
  \prod_{k=1}^{K} \text{InvGamma}(\sigma_k \;;\; a, b).
\end{align*} 

The model in Edward is 
\begin{lstlisting}[language=Python]
class MixtureGaussian:
    """
    Mixture of Gaussians

    p(x, z) = [ prod_{n=1}^N N(x_n; mu_{c_n}, sigma_{c_n}) Multinomial(c_n; pi) ]
              [ prod_{k=1}^K N(mu_k; 0, cI) Inv-Gamma(sigma_k; a, b) ]
              Dirichlet(pi; alpha)

    where z = {pi, mu, sigma} and for known hyperparameters a, b, c, alpha.

    Parameters
    ----------
    K : int
        Number of mixture components.
    D : float, optional
        Dimension of the Gaussians.
    """
    def __init__(self, K, D):
        self.K = K
        self.D = D
        self.num_vars = (2*D + 1) * K

        self.a = 1
        self.b = 1
        self.c = 10
        self.alpha = tf.ones([K])

    def log_prob(self, xs, zs):
        """Returns a vector [log p(xs, zs[1,:]), ..., log p(xs, zs[S,:])]."""
        N = get_dims(xs)[0]
        pi, mus, sigmas = zs
        log_prior = dirichlet.logpdf(pi, self.alpha)
        log_prior += tf.reduce_sum(norm.logpdf(mus, 0, np.sqrt(self.c)), 1)
        log_prior += tf.reduce_sum(invgamma.logpdf(sigmas, self.a, self.b), 1)

        # Loop over each mini-batch zs[b,:]
        log_lik = []
        n_minibatch = get_dims(zs[0])[0]
        for s in range(n_minibatch):
            log_lik_z = N*tf.reduce_sum(tf.log(pi), 1)
            for k in range(self.K):
                log_lik_z += tf.reduce_sum(multivariate_normal.logpdf(xs,
                    mus[s, (k*self.D):((k+1)*self.D)],
                    sigmas[s, (k*self.D):((k+1)*self.D)]))

            log_lik += [log_lik_z]

        return log_prior + tf.pack(log_lik)

model = MixtureGaussian(K=2, D=2)
\end{lstlisting}


\subsubsection{Variational Model}
The latent variables are $\mathbf{z} = (\pi, \mu, \sigma)$.

Define the variational model to be a Dirichlet $\times$ factorized normal
$\times$ factorized inverse gamma
\begin{align*}
  q(\mathbf{z} \;;\; \lambda) 
  &=
  \text{Dirichlet}(\mathbf{z}_\pi) 
  \times
  \mathcal{N}(\mathbf{z}_\mu)
  \times
  \text{InvGamma}(\mathbf{z}_\sigma)
\end{align*}

The model in Edward is
\begin{lstlisting}[language=Python]
variational = Variational()
variational.add(Dirichlet(model.K))
variational.add(Normal(model.K*model.D))
variational.add(InvGamma(model.K*model.D)).
\end{lstlisting}


\subsubsection{Data}

Consider a simulated (pre-generated) dataset
\begin{lstlisting}[language=Python]
x = np.loadtxt('data/mixture_data.txt', dtype='float32', delimiter=',')
data = ed.Data(tf.constant(x, dtype=tf.float32))
\end{lstlisting}


\subsubsection{Inference}

Inference is mean-field variational inference
\begin{lstlisting}[language=Python]
inference = ed.MFVI(model, variational, data)
inference.run(n_iter=500, n_minibatch=5, n_data=5)
\end{lstlisting}
