% Define the subtitle of the page
\title{Gaussian process classification}

% Begin the content of the page
\subsection{Gaussian process classification}

A Gaussian process is a powerful object for modeling nonlinear
relationships between pairs of random variables. It defines a distribution over
(possibly nonlinear) functions, which can be applied for representing
our uncertainty around the true functional relationship.
Here we define a Gaussian process model for classification.

Formally, a distribution over functions $f:\mathbb{R}^D\to\mathbb{R}$ can be specified
by a Gaussian process
\begin{align*}
  p(f)
  &=
  \mathcal{GP}(f; \mathbf{0}, k(\mathbf{x}, \mathbf{x}^\prime)),
\end{align*}
whose mean function is the zero function, and whose covariance
function is some kernel which describes dependence between
any set of inputs to the function.

Given a set of input-output pairs
$\{(\mathbf{x}_n\in\mathbb{R}^D,y_n\in\mathbb{R})\}$,
the likelihood can be written as a multivariate normal
\begin{align*}
  p(\mathbf{y})
  &=
  \mathcal{N}(\mathbf{y} \;;\; \mathbf{0}, \mathbf{K})
\end{align*}
where $\mathbf{K}$ is a covariance matrix given by evaluating
$k(\mathbf{x}, \mathbf{x}^\prime)$ for each pair of inputs in the data
set.

The above applies directly for regression where $\mathbb{y}$ is a
real-valued response, but not for (binary) classification, where $\mathbb{y}$
is a label in $\{0,1\}$. To deal with classification, we interpret the
response as latent variables which is squashed into $[0,1]$. We then
draw from a Bernoulli to determine the label, with probability given
by the squashed value.

Define the likelihood of an observation $(\mathbf{x}_n, y_n)$ as
\begin{align*}
  p(y_n \mid \mathbf{z} \;;\; x_n)
  &=
  \text{Bernoulli}(y_n \mid \text{logit}^{-1}(\mathbf{x}_n^\top \mathbf{z})).
\end{align*}

Define the prior to be a multivariate normal
\begin{align*}
  p(\mathbf{z})
  &=
  \mathcal{N}(\mathbf{z} \;;\; \mathbf{0}, \mathbf{K})
\end{align*}
with
covariance matrix given by evaluating $k(x, x^\prime)$ for each pair of inputs
$(x, x^\prime)$.

Let's build the model in Edward using TensorFlow, using a radial basis function 
(RBF) kernel, also known as the squared exponential or exponentiated quadratic.
\begin{lstlisting}[language=Python]
class GaussianProcess:
    """
    Gaussian process classification

    p((x,y), z) = Bernoulli(y | logit^{-1}(x*z)) *
                  Normal(z | 0, K),

    where z are weights drawn from a GP with covariance given by k(x,
    x') for each pair of inputs (x, x'), and with squared-exponential
    kernel and known kernel hyperparameters.

    Parameters
    ----------
    N : int
        Number of data points.
    sigma : float, optional
        Signal variance parameter.
    l : float, optional
        Length scale parameter.
    """
    def __init__(self, N, sigma=1.0, l=1.0):
        self.N = N
        self.sigma = sigma
        self.l = l

        self.num_vars = N
        self.inverse_link = tf.sigmoid

    def kernel(self, x):
        mat = []
        for i in range(self.N):
            mat += [[]]
            xi = x[i, :]
            for j in range(self.N):
                if j == i:
                    mat[i] += [multivariate_rbf(xi, xi, self.sigma, self.l)]
                else:
                    xj = x[j, :]
                    mat[i] += [multivariate_rbf(xi, xj, self.sigma, self.l)]

            mat[i] = tf.pack(mat[i])

        return tf.pack(mat)

    def log_prob(self, xs, zs):
        """Return a vector [log p(xs, zs[1,:]), ..., log p(xs, zs[S,:])]."""
        x, y = xs['x'], xs['y']
        log_prior = multivariate_normal.logpdf(zs, cov=self.kernel(x))
        log_lik = tf.pack([tf.reduce_sum(
            bernoulli.logpmf(y, self.inverse_link(tf.mul(y, z)))
            ) for z in tf.unpack(zs)])
        return log_prior + log_lik

model = GaussianProcess(N=number_of_datapoints)
\end{lstlisting}

We experiment with this model in the
\href{tut_supervised_classification.html}{supervised learning (classification)} tutorial.

\subsubsection{References}\label{references}

\begin{itemize}
\item
  Rasmussen, C. E., \& Williams, C. K. I. (2006). Gaussian processes
  for machine learning. MIT Press, Cambridge, MA.
\end{itemize}
