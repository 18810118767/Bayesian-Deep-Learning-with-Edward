% Define the subtitle of the page
\title{KL(q||p) minimization}

% Begin the content of the page
\subsection{$\text{KL}(q\|p)$ minimization}

One form of variational inference minimizes the Kullback-Leibler divergence
\textbf{from} $q(z\;;\;\lambda)$ \textbf{to} $p(z \mid x)$,
\begin{align*}
  \lambda^*
  &=
  \arg\min_\lambda \text{KL}(
  q(z\;;\;\lambda)
  \;\|\;
  p(z \mid x)
  )\\
  &=
  \arg\min_\lambda\;
  \mathbb{E}_{q(z\;;\;\lambda)}
  \big[
  \log q(z\;;\;\lambda)
  -
  \log p(z \mid x)
  \big].
\end{align*}
The KL divergence is a non-symmetric, information theoretic measure of
similarity between two probability distributions
\citep{hinton1993keeping,waterhouse1996bayesian,jordan1999introduction}.

\subsubsection{The Evidence Lower Bound}

The above optimization problem is intractable because it directly depends on the
posterior $p(z \mid x)$. To tackle this, consider the property
\begin{align*}
  \log p(x)
  &=
  \text{KL}(
  q(z\;;\;\lambda)
  \;\|\;
  p(z \mid x)
  )\\
  &\quad+\;
  \mathbb{E}_{q(z\;;\;\lambda)}
  \big[
  \log p(x, z)
  -
  \log q(z\;;\;\lambda)
  \big]
\end{align*}
where the left hand side is the logarithm of the marginal likelihood
$p(x) = \int p(x,z) \text{d}z$, also known as the model evidence. (Try
deriving this using Bayes' rule!)

The evidence is a constant with respect to the variational parameters $\lambda$,
so we can minimize $\text{KL}(q\|p)$ by instead maximizing
the Evidence Lower BOund,
\begin{align*}
  \text{ELBO}(\lambda)
  &=\;
  \mathbb{E}_{q(z\;;\;\lambda)}
  \big[
  \log p(x, z)
  -
  \log q(z\;;\;\lambda)
  \big].
\end{align*}
In the ELBO, both $p(x,z)$ and $q(z\;;\;\lambda)$ are
tractable. The optimization problem we seek to solve becomes
\begin{align*}
  \lambda^*
  &=
  \arg \max_\lambda \text{ELBO}(\lambda).
\end{align*}
As per its name, the ELBO is a lower bound on the evidence, and
optimizing it tries to maximize the probability of observing the data.
What does maximizing the ELBO do? Splitting the ELBO reveals a trade-off
\begin{align*}
  \text{ELBO}(\lambda)
  &=\;
  \mathbb{E}_{q(z \;;\; \lambda)}[\log p(x, z)]
  - \mathbb{E}_{q(z \;;\; \lambda)}[\log q(z\;;\;\lambda)],
\end{align*}
where the first term represents an energy and the second term
(including the minus sign) represents the entropy of $q$.
The energy encourages $q$ to focus probability mass where the
model puts high probability, $p(x, z)$.
The entropy encourages $q$ to spread probability mass to avoid
concentrating to one location.

Edward uses two generic strategies to obtain gradients for
optimization.
\begin{itemize}
    \item \href{tut_KLqp_score}{Score function gradient}
    \item \href{tut_KLqp_reparam}{Reparameterization gradient}
  \end{itemize}

\subsubsection{References}\label{references}

