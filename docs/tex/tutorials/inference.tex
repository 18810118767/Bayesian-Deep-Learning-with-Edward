\title{Inference of Probabilistic Models}

\subsection{Inference of Probabilistic Models}

This tutorial asks the question: what does it mean to do inference of
probabilistic models? This sets the stage for understanding how to
design inference algorithms in Edward.

\subsubsection{The posterior}

How can we use a model $p(\mathbf{x}, \mathbf{z})$ to analyze some
data $\mathbf{x}$? In other words, what hidden structure $\mathbf{z}$
explains the data? We seek to infer this hidden structure using the
model.

One method of inference leverages Bayes' rule to define the
\emph{posterior}
\begin{align*}
  p(\mathbf{z} \mid \mathbf{x})
  &=
  \frac{p(\mathbf{x}, \mathbf{z})}{\int p(\mathbf{x}, \mathbf{z}) \text{d}\mathbf{z}}.
\end{align*}
The posterior is the distribution of the latent variables
$\mathbf{z}$, conditioned on some (observed) data $\mathbf{x}$.
Drawing analogy to representation learning, it is a probabilistic
description of the data's hidden representation.

From the perspective of inductivism, as practiced by classical
Bayesians (and implicitly by frequentists),
the posterior is our updated hypothesis about the latent variables.
From the perspective of hypothetico-deductivism, as practiced by
statisticians such as Box, Rubin, and Gelman, the posterior is simply
a fitted model to data, to be criticized and thus revised
\citep{box1982apology,gelman2013philosophy}.

\subsubsection{Inferring the posterior}

Now we know what the posterior represents. How do we calculate it? This is the
central computational challenge in inference.

The posterior is difficult to compute because of its normalizing
constant, which is the integral in the denominator.
This is often a high-dimensional integral that lacks an analytic (closed-form)
solution. Thus, calculating the posterior means \emph{approximating} the
posterior.

For details on how to specify inference in Edward, see the
\href{/api/ed/inferences}{inference API}. We describe several examples in
detail in the \href{/tutorials/}{tutorials}.


\subsubsection{References}\label{references}

