% Define the subtitle of the page
\title{Probability models}

% Begin the content of the page
\subsection{Probability models}

A probabilistic model asserts how observations from a natural phenomenon arise.
The model is a \emph{joint distribution}
\begin{align*}
  p(x, z)
\end{align*}
of observed variables $x$ corresponding to data, and latent
variables $z$ that provide the hidden structure to generate from $x$.
The joint distribution factorizes into two components.

The \emph{likelihood}
\begin{align*}
  p(x \mid z)
\end{align*}
is a probability distribution that describes how any data $x$ depend
on the latent variables $z$. The likelihood posits a data generating
process, where the data $x$ are assumed drawn from the likelihood
conditioned on a particular hidden pattern described by $z$.

The \emph{prior}
\begin{align*}
  p(z)
\end{align*}
is a probability distribution that describes the latent variables
present in the data. It posits a generating process of the hidden structure.

For details on how to specify a model in Edward, see the
\href{/api/models}{model API}. We describe several examples in detail
in the
other model \href{/tutorials/}{tutorials}.
