\title{Gaussian process classification}

\subsection{Gaussian process classification}

A Gaussian process is a powerful object for modeling nonlinear
relationships between pairs of random variables. It defines a distribution over
(possibly nonlinear) functions, which can be applied for representing
our uncertainty around the true functional relationship.
Here we define a Gaussian process model for classification
\citep{rasmussen2006gaussian}.

Formally, a distribution over functions $f:\mathbb{R}^D\to\mathbb{R}$ can be specified
by a Gaussian process
\begin{align*}
  p(f)
  &=
  \mathcal{GP}(f\mid \mathbf{0}, k(\mathbf{x}, \mathbf{x}^\prime)),
\end{align*}
whose mean function is the zero function, and whose covariance
function is some kernel which describes dependence between
any set of inputs to the function.

Given a set of input-output pairs
$\{\mathbf{x}_n\in\mathbb{R}^D,y_n\in\mathbb{R}\}$,
the likelihood can be written as a multivariate normal
\begin{align*}
  p(\mathbf{y})
  &=
  \text{Normal}(\mathbf{y} \mid \mathbf{0}, \mathbf{K})
\end{align*}
where $\mathbf{K}$ is a covariance matrix given by evaluating
$k(\mathbf{x}_n, \mathbf{x}_m)$ for each pair of inputs in the data
set.

The above applies directly for regression where $\mathbb{y}$ is a
real-valued response, but not for (binary) classification, where $\mathbb{y}$
is a label in $\{0,1\}$. To deal with classification, we interpret the
response as latent variables which is squashed into $[0,1]$. We then
draw from a Bernoulli to determine the label, with probability given
by the squashed value.

Define the likelihood of an observation $(\mathbf{x}_n, y_n)$ as
\begin{align*}
  p(y_n \mid \mathbf{z}, x_n)
  &=
  \text{Bernoulli}(y_n \mid \text{logit}^{-1}(\mathbf{x}_n^\top \mathbf{z})).
\end{align*}

Define the prior to be a multivariate normal
\begin{align*}
  p(\mathbf{z})
  &=
  \text{Normal}(\mathbf{z} \mid \mathbf{0}, \mathbf{K}),
\end{align*}
with covariance matrix given as previously stated.

Let's build the model in Edward. We use a radial basis function (RBF)
kernel, also known as the squared exponential or exponentiated
quadratic.
\begin{lstlisting}[language=Python]
from edward.models import Bernoulli, Normal
from edward.util import multivariate_rbf

def kernel(x):
  mat = []
  for i in range(N):
    mat += [[]]
    xi = x[i, :]
    for j in range(N):
      if j == i:
        mat[i] += [multivariate_rbf(xi, xi)]
      else:
        xj = x[j, :]
        mat[i] += [multivariate_rbf(xi, xj)]

    mat[i] = tf.pack(mat[i])

  return tf.pack(mat)

N = 1000  # number of data points
D = 256  # number of features

X = tf.placeholder(tf.float32, [N, D])
f = MultivariateNormalFull(mu=tf.zeros(N), sigma=kernel(X))
y = Bernoulli(logits=f)
\end{lstlisting}

We experiment with this model in the
\href{/tutorials/supervised-classification}{supervised learning (classification)} tutorial.

\subsubsection{References}\label{references}
