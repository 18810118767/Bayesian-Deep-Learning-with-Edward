% Define the subtitle of the page
\title{Reparameterization gradient}

% Begin the content of the page
\subsection{Reparameterization gradient}

(This tutorial follows the
\href{klqp}{$\text{KL}(q\|p)$ minimization} tutorial.)

We seek to maximize the ELBO,
\begin{align*}
  \lambda^*
  &=
  \arg \max_\lambda \text{ELBO}(\lambda)\\
  &=\;
  \arg \max_\lambda\;
  \mathbb{E}_{q(z\;;\;\lambda)}
  \big[
  \log p(x, z)
  -
  \log q(z\;;\;\lambda)
  \big],
\end{align*}
using a ``black box'' algorithm. This means generically inferring the
posterior while making few assumptions about the model.

If the model has differentiable latent variables, then it is generally
advantageous to leverage gradient information from the model in order to
better traverse the optimization space. One approach to doing this is
the reparameterization gradient.

\subsubsection{The reparameterization identity}

Gradient descent is a standard approach for optimizing complicated
objectives like the ELBO. The idea is to calculate its gradient
\begin{align*}
  \nabla_\lambda\;
  \text{ELBO}(\lambda)
  &=
  \nabla_\lambda\;
  \mathbb{E}_{q(z\;;\;\lambda)}
  \big[
  \log p(x, z)
  -
  \log q(z\;;\;\lambda)
  \big],
\end{align*}
and update the current set of parameters proportional to the gradient.

Some variational distributions $q(z\;;\;\lambda)$ admit useful
reparameterizations. For example, we can reparameterize a normal distribution
$z \sim \mathcal{N}(\mu, \Sigma)$ as
$z \sim \mu + L \mathcal{N}(0, I)$ where $\Sigma = LL^\top$. In general, write
this as
\begin{align*}
  \epsilon &\sim q(\epsilon)\\
  z &= z(\epsilon \;;\; \lambda),
\end{align*}
where $\epsilon$ is a random variable that does \textbf{not} depend on the
variational parameters $\lambda$. The deterministic function
$z(\cdot;\lambda)$ encapsulates the variational parameters instead,
and following the process is equivalent to directly drawing $z$ from
the original distribution.

The reparameterization gradient leverages this property of the
variational distribution to write the gradient as
\begin{align*}
  \nabla_\lambda\;
  \text{ELBO}(\lambda)
  &=\;
  \mathbb{E}_{q(\epsilon)}
  \big[
  \nabla_\lambda
  \big(
  \log p(x, z(\epsilon \;;\; \lambda))
  -
  \log q(z(\epsilon \;;\; \lambda) \;;\;\lambda)
  \big)
  \big].
\end{align*}
The gradient of the ELBO is an expectation over the base
distribution $q(\epsilon)$, and the gradient can be applied directly
to the inner expression
\citep{rezende2014stochastic,kingma2014auto}.


Edward uses automatic differentiation, specifically with TensorFlow's
computational graphs, making this gradient computation both simple and
efficient to distribute.

\subsubsection{Noisy estimates using Monte Carlo integration}

We can use Monte Carlo integration to obtain noisy estimates of both the ELBO
and its gradient. The basic procedure follows these steps:
\begin{enumerate}
  \item draw $S$ samples $\{\epsilon_s\}_1^S \sim q(\epsilon)$,
  \item evaluate the argument of the expectation using $\{\epsilon_s\}_1^S$, and
  \item compute the empirical mean of the evaluated quantities.
\end{enumerate}

A Monte Carlo estimate of the gradient is then
\begin{align*}
  \nabla_\lambda\;
  \text{ELBO}(\lambda)
  &\approx\;
  \frac{1}{S}
  \sum_{s=1}^{S}
  \big[
  \nabla_\lambda
  \big(
  \log p(x, z(\epsilon \;;\; \lambda))
  -
  \log q(z(\epsilon \;;\; \lambda) \;;\;\lambda)
  \big)
  \big].
\end{align*}
This is an unbiased estimate of the actual gradient of the ELBO. Empirically, it
exhibits lower variance than the
\href{klqp-score}{score function gradient}, leading to
faster convergence in a large set of problems.

\subsubsection{Implementation}

We implement $\text{KL}(q\|p)$ minimization with the
reparameterization gradient in the class \texttt{MFVI} (mean-field
variational inference). It inherits from
\texttt{VariationalInference}, which
provides a collection of default methods
such as an optimizer.

\begin{lstlisting}[language=Python]
class MFVI(VariationalInference):
  def __init__(self, *args, **kwargs):
    super(MFVI, self).__init__(*args, **kwargs)

  def initialize(self, n_samples=1, ...):
    ...
    self.n_samples = n_samples
    return super(MFVI, self).initialize(*args, **kwargs)

  def build_reparam_loss(self):
    x = self.data
    z = {key: rv.sample([self.n_samples])
         for key, rv in six.iteritems(self.latent_vars)}

    p_log_prob = self.model_wrapper.log_prob(x, z)
    q_log_prob = 0.0
    for key, rv in six.iteritems(self.latent_vars):
      q_log_prob += tf.reduce_sum(rv.log_prob(z[key]),
                                  list(range(1, len(rv.get_batch_shape()) + 1)))

    self.loss = tf.reduce_mean(p_log_prob - q_log_prob)
    return -self.loss
\end{lstlisting}

Two methods are added: \texttt{initialize()} and
\texttt{build_reparam_loss()}. The method \texttt{initialize()}
follows the same initialization as \texttt{VariationalInference}, and
adds an argument: \texttt{n_samples} for the number of samples from
the variational model.

The method \texttt{build_reparam_loss()} implements the ELBO and its
gradient. It draws \texttt{self.n_samples} samples from the
variational model. It registers the Monte Carlo
estimate of the ELBO in TensorFlow's computational graph, and stores it
in \texttt{self.loss}, to track progress of the inference for diagnostics.

The method returns an object whose automatic differentiation is the
reparameterization gradient of the ELBO. The computational graph will
simply traverse the nodes during backpropagation; unlike the
\href{klqp-score}{score function gradient}, the
reparameterization gradient does not need TensorFlow's
\texttt{tf.stop_gradient()} function.

There is a nuance here. TensorFlow's optimizers are configured to
\emph{minimize} an objective function. So the gradient is set to be
the negative of the ELBO's gradient.

See the \href{../api/index.html}{API} for more details.

\subsubsection{References}\label{references}

