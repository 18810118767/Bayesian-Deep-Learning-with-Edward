\title{Bayesian Linear Regression}

\subsection{Bayesian Linear Regression}

Bayesian linear regression posits a model of
outputs $y\in\mathbb{R}$, also known as the response, given
a vector of inputs
$\mathbf{x}\in\mathbb{R}^D$, also known as the features or covariates.
The model assumes a
linear relationship between these two random variables
\citep{murphy2012machine}.

For a set of $N$ data points $(\mathbf{X},\mathbf{y})=\{(\mathbf{x}_n, y_n)\}$,
the model posits the following distributions:
\begin{align*}
  p(\mathbf{w})
  &=
  \text{Normal}(\mathbf{w} \mid \mathbf{0}, \sigma_w^2\mathbf{I}),
  \\[1.5ex]
  p(b)
  &=
  \text{Normal}(b \mid 0, \sigma_b^2),
  \\
  p(\mathbf{y} \mid \mathbf{w}, b, \mathbf{X})
  &=
  \prod_{n=1}^N
  \text{Normal}(y_n \mid \mathbf{x}_n^\top\mathbf{w} + b, \sigma_y^2).
\end{align*}
The latent variables are the linear model's weights $\mathbf{w}$ and
intercept $b$, also known as the bias.
Assume $\sigma_w^2,\sigma_b^2$ are known prior variances and $\sigma_y^2$ is a
known likelihood variance. The mean of the likelihood is given by a
linear transformation of the inputs $\mathbf{x}_n$.

Let's build the model in Edward, fixing $\sigma_w,\sigma_b,\sigma_y=1$.
\begin{lstlisting}[language=Python]
from edward.models import Normal

N = 40  # number of data points
D = 10  # number of features

X = tf.placeholder(tf.float32, [N, D])
w = Normal(mu=tf.zeros(D), sigma=tf.ones(D))
b = Normal(mu=tf.zeros(1), sigma=tf.ones(1))
y = Normal(mu=ed.dot(X, w) + b, sigma=tf.ones(N))
\end{lstlisting}
Here, we define a placeholder \texttt{X}. During inference, we pass in
the value for this placeholder according to data.

We experiment with this model in the \href{/tutorials/supervised-regression}{supervised
learning (regression)} tutorial.
An example script is available
\href{https://github.com/blei-lab/edward/blob/master/examples/bayesian_linear_regression.py}
{here}.

\subsubsection{References}\label{references}
