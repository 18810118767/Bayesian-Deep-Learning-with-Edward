\title{Score function gradient}

\subsection{Score function gradient}

(This tutorial follows the
\href{/tutorials/klqp}{$\text{KL}(q\|p)$ minimization} tutorial.)

We seek to maximize the ELBO,
\begin{align*}
  \lambda^*
  &=
  \arg \max_\lambda \text{ELBO}(\lambda)\\
  &=\;
  \arg \max_\lambda\;
  \mathbb{E}_{q(z\;;\;\lambda)}
  \big[
  \log p(x, z)
  -
  \log q(z\;;\;\lambda)
  \big],
\end{align*}
using a ``black box'' algorithm. This means generically inferring the
posterior while making few assumptions about the model.

\subsubsection{The score function identity}

Gradient descent is a standard approach for optimizing complicated
objectives like the ELBO. The idea is to calculate its gradient
\begin{align*}
  \nabla_\lambda\;
  \text{ELBO}(\lambda)
  &=
  \nabla_\lambda\;
  \mathbb{E}_{q(z\;;\;\lambda)}
  \big[
  \log p(x, z)
  -
  \log q(z\;;\;\lambda)
  \big],
\end{align*}
and update the current set of parameters proportional to the gradient.

The score function gradient estimator leverages a property of
logarithms to write the gradient as
\begin{align*}
  \nabla_\lambda\;
  \text{ELBO}(\lambda)
  &=\;
  \mathbb{E}_{q(z\;;\;\lambda)}
  \big[
  \nabla_\lambda \log q(z\;;\;\lambda)
  \:
  \big(
  \log p(x, z)
  -
  \log q(z\;;\;\lambda)
  \big)
  \big].
\end{align*}
The gradient of the ELBO is an expectation over the variational
model $q(z\;;\;\lambda)$; the only new ingredient it requires is the
\emph{score function} $\nabla_\lambda \log q(z\;;\;\lambda)$
\citep{paisley2012variational,ranganath2014black}.

Edward uses automatic differentiation, specifically with TensorFlow's
computational graphs, making this gradient computation both simple and
efficient to distribute.

\subsubsection{Noisy estimates using Monte Carlo integration}

We can use Monte Carlo integration to obtain noisy estimates of both the ELBO
and its gradient. The basic procedure follows these steps:
\begin{enumerate}
  \item draw $S$ samples $\{z_s\}_1^S \sim q(z\;;\;\lambda)$,
  \item evaluate the argument of the expectation using $\{z_s\}_1^S$, and
  \item compute the empirical mean of the evaluated quantities.
\end{enumerate}

A Monte Carlo estimate of the gradient is then
\begin{align*}
  \nabla_\lambda\;
  \text{ELBO}(\lambda)
  &\approx\;
  \frac{1}{S}
  \sum_{s=1}^{S}
  \big[
  \big(
  \log p(x, z_s)
  -
  \log q(z_s\;;\;\lambda)
  \big)
  \:
  \nabla_\lambda \log q(z_s\;;\;\lambda)
  \big].
\end{align*}
This is an unbiased estimate of the actual gradient of the ELBO.

\subsubsection{Implementation}

\textbf{Note: These details are outdated since Edward v1.1.3.}

We implement $\text{KL}(q\|p)$ minimization with the score function
gradient in the class \texttt{KLqp}.
It inherits from \texttt{VariationalInference}, which
provides a collection of default
methods such as an optimizer.

\begin{lstlisting}[language=Python]
class KLqp(VariationalInference):
  def __init__(self, *args, **kwargs):
    super(KLqp, self).__init__(*args, **kwargs)

  def initialize(self, n_samples=1, ...):
    ...
    self.n_samples = n_samples
    return super(KLqp, self).initialize(*args, **kwargs)

  def build_score_loss(self):
    x = self.data
    z = {key: rv.sample([self.n_samples])
         for key, rv in six.iteritems(self.latent_vars)}

    p_log_prob = self.model_wrapper.log_prob(x, z)
    q_log_prob = 0.0
    for key, rv in six.iteritems(self.latent_vars):
      q_log_prob += tf.reduce_sum(rv.log_prob(tf.stop_gradient(z[key])),
                                  list(range(1, len(rv.get_batch_shape()) + 1)))

    losses = p_log_prob - q_log_prob
    self.loss = -tf.reduce_mean(losses)
    return -tf.reduce_mean(q_log_prob * tf.stop_gradient(losses))
\end{lstlisting}

Two methods are added: \texttt{initialize()} and
\texttt{build_score_loss()}.  The method \texttt{initialize()} follows
the same initialization as \texttt{VariationalInference}, and adds an
argument: \texttt{n_samples} for the number of samples from the
variational model.

The method \texttt{build_score_loss()} implements the ELBO and its
gradient. It draws \texttt{self.n_samples} samples from the
variational model. It registers the Monte Carlo
estimate of the ELBO in TensorFlow's computational graph, and stores it
negative in \texttt{self.loss}, to track progress of the inference for diagnostics.

The method returns an object whose automatic differentiation is the
score function gradient of the ELBO. The TensorFlow function
\texttt{tf.stop_gradient()} tells the computational graph to stop
traversing nodes to propagate gradients. In this case,
the only gradients taken are $\nabla_\lambda \log q(z_s\;;\;\lambda)$,
one for each sample $z_s$. We multiply it by \texttt{losses}
element-wise and return the mean.

There is a nuance here. TensorFlow's optimizers are configured to
\emph{minimize} an objective function. So the gradient is set to be
the negative of the ELBO's gradient.

See the \href{/api/}{API} for more details.

\subsubsection{References}\label{references}
