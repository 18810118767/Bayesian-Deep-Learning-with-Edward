\title{Data}

{{navbar}}

\subsubsection{Data}

Data defines a set of observations. In Edward, data is represented as
TensorFlow tensors or NumPy arrays.

\begin{lstlisting}[language=Python]
x_data = np.array([0, 1, 0, 0, 0, 0, 0, 0, 0, 1])
x_data = tf.constant([0, 1, 0, 0, 0, 0, 0, 0, 0, 1])
\end{lstlisting}

Edward can also work with batch tensors for settings when the full
data does not fit in memory.

In general, there are three ways to read data in Edward. (They follow
the \href{https://www.tensorflow.org/versions/master/how_tos/reading_data/index.html}
{three ways to read data in TensorFlow}.)

\begin{enumerate}
\item
   \textbf{Preloaded data.} A constant or variable in the TensorFlow graph
   holds all the data.

   This setting is the fastest to work with and is recommended if the
   data fits in memory.

   Represent the data as NumPy arrays.
   Internally, during inference, we will store them in TensorFlow variables to prevent
   copying data more than once in memory.
   (As an example, see
   the
   \href{https://github.com/blei-lab/edward/blob/master/examples/tf_mixture_gaussian.py}
   {mixture of Gaussians}.)
\item
   \textbf{Feeding.} Manual code provides the data when running each step of
   inference.

   This setting provides the most fine-grained control which is useful for experimentation.

   Represent the data as TensorFlow placeholders. During inference,
   the user must manually feed the placeholders at each
   step by first initializing via \texttt{inference.initialize()}; then
   in a loop call \texttt{inference.update(feed_dict={...})} where
   \texttt{feed_dict} carries the values for the \texttt{tf.placeholder}'s.
   (As an example, see
   the
   \href{https://github.com/blei-lab/edward/blob/master/examples/bayesian_linear_regression.py}
   {Bayesian linear regression}.)
\item
   \textbf{Reading from files.} An input pipeline reads the data from files
   at the beginning of a TensorFlow graph.

   This setting is recommended if the data does not fit in memory.

   Represent the data as TensorFlow tensors, where the tensors are the
   output of data readers. During inference, each update will be
   automatically evaluated over new batch tensors represented through
   the data readers. (As an example, see
   the
   \href{https://github.com/blei-lab/edward/blob/master/tests/test_inference_data.py}
   {data unit test}.)
\end{enumerate}

\subsubsection{Training Models with Data}

To pass in data during inference, we form a Python dictionary. Each
item in the dictionary has a random variable binded to the data
values.

\begin{lstlisting}[language=Python]
# assuming `x` and `y` form observed variables in the model
data = {x: x_data, y: y_data}
\end{lstlisting}


How do we use the data during training? In general there are three use
cases:

\begin{enumerate}
\item
\textbf{Train over the full data per step.}

Follow the setting of preloaded data.
\item
\textbf{Train over a batch per step when the full data fits in memory.} This
scales inference in terms of computational complexity.

Follow the setting of preloaded data. Specify the batch size with
\texttt{n_minibatch} in \texttt{Inference}. By default, we will subsample by
slicing along the first dimension of every data structure in the
data dictionary. Alternatively, follow the setting of feeding.
Manually deal with the batch behavior at each training step.
\item
\textbf{Train over batches per step when the full data does not fit in
memory.} This scales inference in terms of computational complexity and
memory complexity.

Follow the setting of reading from files. Alternatively, follow the
setting of feeding, and use a generator to create and destroy NumPy
arrays on the fly for feeding the placeholders.
\end{enumerate}
