\title{Deep Probabilistic Programming}

\subsection{Deep Probabilistic Programming}

This webpage is a companion to the article,
\href{https://arxiv.org/abs/1701.03757}{Deep Probabilistic Programming}
\citep{tran2017deep}.
Here we provide more details for plug-and-play with the code snippets.
An interactive version with Jupyter notebook is available
\href{http://nbviewer.jupyter.org/github/blei-lab/edward/blob/master/notebooks/iclr2017.ipynb}{here}.

The code snippets assume the following versions.

\begin{lstlisting}[language=bash]
pip install edward==1.2.3
pip install tensorflow==1.0.0  # alternatively, tensorflow-gpu==1.0.0
pip install keras==1.0.0
\end{lstlisting}

\subsubsection{Section 3. Compositional Representations for Probabilistic Models}

\textbf{Figure 1}. Beta-Bernoulli program.
\begin{lstlisting}[language=python]
import tensorflow as tf
from edward.models import Bernoulli, Beta

theta = Beta(a=1.0, b=1.0)
x = Bernoulli(p=tf.ones(50) * theta)
\end{lstlisting}
For an example of it in use, see
\href{https://github.com/blei-lab/edward/blob/master/examples/beta_bernoulli.py}{\texttt{examples/beta_bernoulli.py}}
in the Github repository.

\textbf{Figure 2}. Variational auto-encoder for a data set of 28 x 28 pixel images
\citep{kingma2014auto,rezende2014stochastic}.
\begin{lstlisting}[language=python]
import tensorflow as tf
from edward.models import Bernoulli, Normal
from keras.layers import Dense

N = 55000  # number of data points
d = 50  # latent dimension

# Probabilistic model
z = Normal(mu=tf.zeros([N, d]), sigma=tf.ones([N, d]))
h = Dense(256, activation='relu')(z)
x = Bernoulli(logits=Dense(28 * 28, activation=None)(h))

# Variational model
qx = tf.placeholder(tf.float32, [N, 28 * 28])
qh = Dense(256, activation='relu')(qx)
qz = Normal(mu=Dense(d, activation=None)(qh),
            sigma=Dense(d, activation='softplus')(qh))
\end{lstlisting}
For an example of it in use, see
\href{https://github.com/blei-lab/edward/blob/master/examples/vae.py}{\texttt{examples/vae.py}}
in the Github repository.

\textbf{Figure 3}. Bayesian recurrent neural network \citep{neal2012bayesian}.
The program has an unspecified number of time steps; it uses a
symbolic for loop (\texttt{tf.scan}).
\begin{lstlisting}[language=python]
import edward as ed
import tensorflow as tf
from edward.models import Normal

H = 50  # number of hidden units
D = 10  # number of features

def rnn_cell(hprev, xt):
  return tf.tanh(ed.dot(hprev, Wh) + ed.dot(xt, Wx) + bh)

Wh = Normal(mu=tf.zeros([H, H]), sigma=tf.ones([H, H]))
Wx = Normal(mu=tf.zeros([D, H]), sigma=tf.ones([D, H]))
Wy = Normal(mu=tf.zeros([H, 1]), sigma=tf.ones([H, 1]))
bh = Normal(mu=tf.zeros(H), sigma=tf.ones(H))
by = Normal(mu=tf.zeros(1), sigma=tf.ones(1))

x = tf.placeholder(tf.float32, [None, D])
h = tf.scan(rnn_cell, x, initializer=tf.zeros(H))
y = Normal(mu=tf.matmul(h, Wy) + by, sigma=1.0)
\end{lstlisting}

\subsubsection{Section 4. Compositional Representations for Inference}

\textbf{Figure 5}. Hierarchical model \citep{gelman2006data}.
  It is a mixture of Gaussians over
  $D$-dimensional data $\{x_n\}\in\mathbb{R}^{N\times D}$. There are
  $K$ latent cluster means $\beta\in\mathbb{R}^{K\times D}$.
\begin{lstlisting}[language=python]
import tensorflow as tf
from edward.models import Categorical, Normal

N = 10000  # number of data points
D = 2  # data dimension
K = 5  # number of clusters

beta = Normal(mu=tf.zeros([K, D]), sigma=tf.ones([K, D]))
z = Categorical(logits=tf.zeros([N, K]))
x = Normal(mu=tf.gather(beta, z), sigma=tf.ones([N, D]))
\end{lstlisting}
It is used below in Figure 6 (left/right), Figure * (variational EM),
and Figure * (data subsampling).

\textbf{Figure 6} \textbf{(left)}. Variational inference
\citep{jordan1999introduction}.
It performs inference on the model defined in Figure 5.
\begin{lstlisting}[language=python]
import edward as ed
import numpy as np
import tensorflow as tf
from edward.models import Categorical, Normal

x_train = np.zeros([N, D], dtype=np.float32)

qbeta = Normal(mu=tf.Variable(tf.zeros([K, D])),
               sigma=tf.exp(tf.Variable(tf.zeros([K, D]))))
qz = Categorical(logits=tf.Variable(tf.zeros([N, K])))

inference = ed.VariationalInference({beta: qbeta, z: qz}, data={x: x_train})
\end{lstlisting}

\textbf{Figure 6} \textbf{(right)}. Monte Carlo \citep{robert1999monte}.
It performs inference on the model defined in Figure 5.
\begin{lstlisting}[language=python]
import edward as ed
import numpy as np
import tensorflow as tf
from edward.models import Empirical

x_train = np.zeros([N, D], dtype=np.float32)

T = 10000  # number of samples
qbeta = Empirical(params=tf.Variable(tf.zeros([T, K, D])))
qz = Empirical(params=tf.Variable(tf.zeros([T, N])))

inference = ed.MonteCarlo({beta: qbeta, z: qz}, data={x: x_train})
\end{lstlisting}

\textbf{Figure 7}. Generative adversarial network
\citep{goodfellow2014generative}.
\begin{lstlisting}[language=python]
import edward as ed
import numpy as np
import tensorflow as tf
from edward.models import Normal
from keras.layers import Dense

N = 55000  # number of data points
d = 50  # latent dimension

def generative_network(eps):
  h = Dense(256, activation='relu')(eps)
  return Dense(28 * 28, activation=None)(h)

def discriminative_network(x):
  h = Dense(28 * 28, activation='relu')(x)
  return Dense(1, activation=None)(h)

# DATA
x_train = np.zeros([N, 28 * 28], dtype=np.float32)

# MODEL
eps = Normal(mu=tf.zeros([N, d]), sigma=tf.ones([N, d]))
x = generative_network(eps)

# INFERENCE
inference = ed.GANInference(data={x: x_train},
    discriminator=discriminative_network)
\end{lstlisting}
For an example of it in use, see
\href{https://github.com/blei-lab/edward/blob/master/examples/gan.py}{\texttt{examples/gan.py}}
in the Github repository.

\textbf{Figure *}. Variational EM \citep{neal1993new}.
It performs inference on the model defined in Figure 5.
\begin{lstlisting}[language=python]
import edward as ed
import numpy as np
import tensorflow as tf
from edward.models import Categorical, PointMass

# DATA
x_train = np.zeros([N, D], dtype=np.float32)

# INFERENCE
qbeta = PointMass(params=tf.Variable(tf.zeros([K, D])))
qz = Categorical(logits=tf.Variable(tf.zeros([N, K])))

inference_e = ed.VariationalInference({z: qz}, data={x: x_train, beta: qbeta})
inference_m = ed.MAP({beta: qbeta}, data={x: x_train, z: qz})

inference_e.initialize()
inference_m.initialize()

tf.initialize_all_variables().run()

for _ in range(10000):
  inference_e.update()
  inference_m.update()
\end{lstlisting}
For more details, see the
\href{/api/inference-compositionality}{inference compositionality} webpage.
See
\href{https://github.com/blei-lab/edward/blob/master/examples/factor_analysis.py}{\texttt{examples/factor_analysis.py}} for
a version performing Monte Carlo EM for logistic factor analysis
in the Github repository.
It leverages Hamiltonian Monte Carlo for the E-step to perform maximum
marginal a posteriori.

\textbf{Figure *}. Data subsampling.
It performs inference on the model defined in Figure 5.
\begin{lstlisting}[language=python]
import edward as ed
import tensorflow as tf
from edward.models import Categorical, Normal

N = 10000  # number of data points
M = 128  # batch size during training
D = 2  # data dimension
K = 5  # number of clusters

# DATA
x_batch = tf.placeholder(tf.float32, [M, D])

# MODEL
beta = Normal(mu=tf.zeros([K, D]), sigma=tf.ones([K, D]))
z = Categorical(logits=tf.zeros([M, K]))
x = Normal(mu=tf.gather(beta, z), sigma=tf.ones([M, D]))

# INFERENCE
qbeta = Normal(mu=tf.Variable(tf.zeros([K, D])),
               sigma=tf.nn.softplus(tf.Variable(tf.zeros([K, D]))))
qz = Categorical(logits=tf.Variable(tf.zeros([M, D])))

inference = ed.VariationalInference({beta: qbeta, z: qz}, data={x: x_batch})
inference.initialize(scale={x: float(N) / M, z: float(N) / M})
\end{lstlisting}
For more details, see the
\href{/api/inference-data-subsampling}{data subsampling} webpage.

\subsubsection{Section 5. Experiments}

\textbf{Figure 9}. Bayesian logistic regression with Hamiltonian Monte Carlo.
\begin{lstlisting}[language=python]
import edward as ed
import numpy as np
import tensorflow as tf
from edward.models import Bernoulli, Empirical, Normal

N = 581012  # number of data points
D = 54  # number of features
T = 100  # number of empirical samples

# DATA
x_data = np.zeros([N, D], dtype=np.float32)
y_data = np.zeros([N], dtype=np.float32)

# MODEL
x = tf.Variable(x_data, trainable=False)
beta = Normal(mu=tf.zeros(D), sigma=tf.ones(D))
y = Bernoulli(logits=ed.dot(x, beta))

# INFERENCE
qbeta = Empirical(params=tf.Variable(tf.zeros([T, D])))
inference = ed.HMC({beta: qbeta}, data={y: y_data})
inference.run(step_size=0.5 / N, n_steps=10)
\end{lstlisting}
For an example of it in use, see
\href{https://github.com/blei-lab/edward/blob/master/examples/bayesian_logistic_regression.py}{\texttt{examples/bayesian_logistic_regression.py}}
in the Github repository.

\subsubsection{Appendix A. Model Examples}

\textbf{Figure 10}. Bayesian neural network for classification \citep{denker1987large}.
\begin{lstlisting}[language=python]
import tensorflow as tf
from edward.models import Bernoulli, Normal

N = 1000  # number of data points
D = 528  # number of features
H = 256  # hidden layer size

W_0 = Normal(mu=tf.zeros([D, H]), sigma=tf.ones([D, H]))
W_1 = Normal(mu=tf.zeros([H, 1]), sigma=tf.ones([H, 1]))
b_0 = Normal(mu=tf.zeros(H), sigma=tf.ones(H))
b_1 = Normal(mu=tf.zeros(1), sigma=tf.ones(1))

x = tf.placeholder(tf.float32, [N, D])
y = Bernoulli(logits=tf.matmul(tf.nn.tanh(tf.matmul(x, W_0) + b_0), W_1) + b_1)
\end{lstlisting}
For an example of it in use, see
\href{https://github.com/blei-lab/edward/blob/master/examples/getting_started_example.py}{\texttt{examples/getting_started_example.py}}
in the Github repository.

\textbf{Figure 11}. Latent Dirichlet allocation \citep{blei2003latent}.
\begin{lstlisting}[language=python]
import tensorflow as tf
from edward.models import Categorical, Dirichlet

D = 4  # number of documents
N = [11502, 213, 1523, 1351]  # words per doc
K = 10  # number of topics
V = 100000  # vocabulary size

theta = Dirichlet(alpha=tf.zeros([D, K]) + 0.1)
phi = Dirichlet(alpha=tf.zeros([K, V]) + 0.05)
z = [[0] * N] * D
w = [[0] * N] * D
for d in range(D):
  for n in range(N[d]):
    z[d][n] = Categorical(pi=theta[d, :])
    w[d][n] = Categorical(pi=phi[z[d][n], :])
\end{lstlisting}

\textbf{Figure 12}. Gaussian matrix factorization
\citep{salakhutdinov2011probabilistic}.
\begin{lstlisting}[language=python]
import tensorflow as tf
from edward.models import Normal

N = 10
M = 10
K = 5  # latent dimension

U = Normal(mu=tf.zeros([M, K]), sigma=tf.ones([M, K]))
V = Normal(mu=tf.zeros([N, K]), sigma=tf.ones([N, K]))
Y = Normal(mu=tf.matmul(U, V, transpose_b=True), sigma=tf.ones([N, M]))
\end{lstlisting}

\textbf{Figure 13}. Dirichlet process mixture model \citep{antoniak1974mixtures}.
\begin{lstlisting}[language=python]
import tensorflow as tf
from edward.models import Normal

mu = DirichletProcess(  # see script for implementation
    alpha=0.1, base_cls=Normal, mu=tf.zeros(D), sigma=tf.ones(D), sample_n=N)
x = Normal(mu=mu, sigma=tf.ones([N, D]))
\end{lstlisting}
To see the essential component defining the \texttt{DirichletProcess}
random variable, see
\href{https://github.com/blei-lab/edward/blob/master/examples/pp_dirichlet_process.py}{\texttt{examples/pp_dirichlet_process.py}}
in the Github repository.
A more involved version with a base distribution is available at
\href{https://github.com/blei-lab/edward/blob/master/examples/pp_dirichlet_process_base.py}{\texttt{examples/pp_dirichlet_process_base.py}}
in the Github repository.

\subsubsection{Appendix B. Inference Examples}

\textbf{Figure *}. Stochastic variational inference \citep{hoffman2013stochastic}.
For more details, see the
\href{/api/inference-data-subsampling}{data subsampling} webpage.

\subsubsection{Appendix C. Complete Examples}

\textbf{Figure 15}. Variational auto-encoder
\citep{kingma2014auto,rezende2014stochastic}.
See the script
\href{https://github.com/blei-lab/edward/blob/master/examples/vae.py}{\texttt{examples/vae.py}}
in the Github repository.

\textbf{Figure 16}. Exponential family embedding \citep{rudolph2016exponential}.
A Github repository with comprehensive features is available at
\href{https://github.com/mariru/exponential_family_embeddings}{mariru/exponential_family_embeddings}.

\subsubsection{References}\label{references}
