\title{Contributing}

\subsection{Contributing}

Interested in contributing to Edward? We appreciate all kinds of help!

\subsubsection{Bug reports, feature requests, user-related questions}

Report any bugs in Edward's
\href{https://github.com/blei-lab/edward/issues}
{Github issues}.
We also appreciate feature requests so that we can prioritize
development according to the users.
We plan to have a
mailing list for Edward users in the future,
but for now also submit usage questions there.

\subsubsection{Pull requests}

We gladly welcome pull requests.

Before making any changes, we recommend opening an issue (if it
doesn't already exist) and discussing your proposed changes. This will
let us give you feedback on whether the changes should be made and
advice on how they should be made.
If the changes are minor, then feel free to make them
without discussion.

Want to contribute but not sure what to contribute? Here are two
suggestions:
\begin{enumerate}
\item
Add a new example or tutorial.
New scripts in the
\href{https://github.com/blei-lab/edward/tree/master/examples}
{\texttt{examples/}} directory are easy to add. They can also
get you acquainted with Edward's developer process.
Alternatively, add new tutorials to the
\href{tutorials}{website} from the
\href{https://github.com/blei-lab/edward/tree/master/docs/tex}
{\texttt{docs/tex/}} directory.
\item
Solve one of Edward's existing
\href{https://github.com/blei-lab/edward/issues}{Github issues}.
They range from low-level software bugs to higher-level design problems.
One or two may pique your interest!
\end{enumerate}

To make changes,
\href{https://help.github.com/articles/working-with-forks/}{fork the repo}.
Write code in this fork with the following guidelines in mind:

\textbf{Style guide.}
Follow
\href{https://www.tensorflow.org/versions/master/how_tos/style_guide.html}{TensorFlow's
style guide}
for writing code, which more or less follows PEP 8.
Follow
\href{https://github.com/numpy/numpy/blob/master/doc/HOWTO_DOCUMENT.rst.txt}
{NumPy's documentation guide}
for writing documention,
and follow
\href{https://www.tensorflow.org/versions/master/how_tos/documentation/index.html}{TensorFlow's documentation guide}
for describing TensorFlow tensors in the documentation.

\textbf{Unit testing.}
Unit testing is awesome. It's useful not only
for checking code after having written it, but also in checking code
as you are developing it.
Most pull requests require unit tests.
Follow
\href{https://www.tensorflow.org/versions/master/api_docs/python/test.html}
{TensorFlow's testing guide}.
To run the unit tests, we use
\href{http://doc.pytest.org/}{py.test}:
run \texttt{py.test tests/}.
To run specific unit tests, call \texttt{py.test} individually, e.g., \texttt{py.test tests/test_metrics.py}.

Submit the pull request whenever you're ready. You can also submit an
incomplete pull request to get intermediate feedback.

\subsubsection{Suggested developer workflow}\label{suggested-workflow}

If you're developing in Edward,
we recommend downloading and installing Edward locally. This enables
any local changes you make to Edward to appear any time you
import Edward in a Python session. To install locally, run the
following:

\begin{verbatim}
git clone git@github.com:blei-lab/edward.git
pip install -e edward
\end{verbatim}

We recommend not installing with \texttt{sudo} but with
\texttt{virtualenv}. In fact, we recommend
\texttt{virtualenvwrapper}, which is a light wrapper on top of
\texttt{virtualenv}.
\href{http://docs.python-guide.org/en/latest/starting/install/osx/}
{Here is a guide on how to set it up}.

All code should be compatible with both Python 2 and 3.
\texttt{virtualenvwrapper} makes it easy to develop in both Python versions
and let them manage their own packages.
If you set up \texttt{virtualenvwrapper} from the above guide,
\href{http://www.marinamele.com/2014/07/install-python3-on-mac-os-x-and-use-virtualenv-and-virtualenvwrapper.html}
{here is a guide on how to setup a virtualenv with Python 3}.
Now we need only work on the corresponding virtualenv to ensure Python
2 and 3 compatibility.

\subsubsection{Suggested private developer workflow}
\label{suggested-private-workflow}

Sometimes it's useful to work privately on Edward, such as for
developing unpublished experiments.
To do this, we suggest using a private repo
that maintains the master branch from the Edward repo. Development
happens on the private repo's branch, and when it is finished, you can
push it to a public repo's branch and then submit a pull request. We
describe this in detail.

Clone the private repo so you can work on it (create a repo if it does
not exist).

\begin{lstlisting}[class=JSON]
git clone https://github.com/blei-lab/edward-private.git
\end{lstlisting}

Pull changes from the public repo. This will let the private repo have
the latest code from the public repo on its master branch.

\begin{lstlisting}[class=JSON]
cd edward-private
git remote add public https://github.com/blei-lab/edward.git
git pull public master # Creates a merge commit
git push origin master
\end{lstlisting}

Now create your branch on the private repo, develop stuff, and pull any
latest changes from the public repo as you develop
(\texttt{git\ pull\ public\ master}).

Finally, to create a pull request from a private repo's branch to the
public repo, push the private branch to the public repo.

\begin{lstlisting}[class=JSON]
git clone https://github.com/blei-lab/edward.git
cd edward
git remote add private https://github.com/blei-lab/edward-private.git
git checkout -b pull_request_yourname
git pull private master
git push origin pull_request_yourname
\end{lstlisting}

Now simply create a pull request via the Github UI on the public repo.
Once project owners review the pull request, they can merge it.
\href{http://stackoverflow.com/questions/10065526/github-how-to-make-a-fork-of-public-repository-private/30352360\#30352360}{Source}

\subsubsection{Internal}

\begin{itemize}
\item
  For developers with push permission to Edward's repo, see
  \href{https://github.com/stan-dev/stan/wiki/Developer-Process#information-to-include-in-pull-request}{Stan's
  process} for how to name branches. Do not merge your own pull
  requests or ever push to master. Someone should always review your
  code. After merging (or deciding to close the request without
  merging), always delete the branch from the repo.
\item
  \textbf{Issue labeling system.} We use
  \href{https://github.com/stan-dev/stan/issues}{Stan's labeling system}.
  While several labels obviously don't apply to us, it's better than the
  default labels.
\item
  \textbf{Packaging and submitting to PyPI.} First, update the version
  number in \texttt{version.py}. Second, follow
  \href{https://packaging.python.org/en/latest/distributing/\#packaging-your-project}{these
  steps}. For shorthand, the sequence of commands is

\begin{lstlisting}[class=JSON]
python setup.py sdist
python setup.py bdist_wheel
twine upload dist/*
\end{lstlisting}

  Third, tag the release on Github and note the new additions when
  tagging this release. You can do this by comparing commits from the
  previous tagged release to master. A link that compares tagged commits
  to master is available on the
  \href{https://github.com/blei-lab/edward/releases}{releases page}.
\end{itemize}
