\title{Bayesian neural network}

\subsection{Bayesian neural network}

A Bayesian neural network is a neural network with a prior
distribution on its weights \citep{neal2012bayesian}.

Consider a data set $\{(\mathbf{x}_n, y_n)\}$, where each data point
comprises of features $\mathbf{x}_n\in\mathbb{R}^D$ and output
$y_n\in\mathbb{R}$. Define the likelihood as
\begin{align*}
  p(y_n \mid \mathbf{z}, \mathbf{x}_n, \sigma^2)
  &=
  \text{Normal}(y_n \mid \mathrm{NN}(\mathbf{x}_n\;;\;\mathbf{z}), \sigma^2),
\end{align*}
where $\mathrm{NN}$ is a neural network whose weights and biases form
the latent variables $\mathbf{z}$. Assume $\sigma^2$ is a
known variance.

Define the prior on the weights and biases $\mathbf{z}$ to be the standard normal
\begin{align*}
  p(\mathbf{z})
  &=
  \text{Normal}(\mathbf{z} \mid \mathbf{0}, \mathbf{I}).
\end{align*}

Let's build the model in Edward. We define a 3-layer Bayesian neural
network with $\tanh$ nonlinearities.
\begin{lstlisting}[language=Python]
from edward.models import Normal

def neural_network(x):
    h = tf.tanh(tf.matmul(x, W_0) + b_0)
    h = tf.tanh(tf.matmul(h, W_1) + b_1)
    h = tf.matmul(h, W_2) + b_2
    return tf.reshape(h, [-1])

N = 40  # number of data ponts
D = 1   # number of features

W_0 = Normal(mu=tf.zeros([D, 10]), sigma=tf.ones([D, 10]))
W_1 = Normal(mu=tf.zeros([10, 10]), sigma=tf.ones([10, 10]))
W_2 = Normal(mu=tf.zeros([10, 1]), sigma=tf.ones([10, 1]))
b_0 = Normal(mu=tf.zeros(10), sigma=tf.ones(10))
b_1 = Normal(mu=tf.zeros(10), sigma=tf.ones(10))
b_2 = Normal(mu=tf.zeros(1), sigma=tf.ones(1))

x = tf.convert_to_tensor(x_train, dtype=tf.float32)
y = Normal(mu=neural_network(x), sigma=0.1 * tf.ones(N))
\end{lstlisting}
This code builds the model assuming the features \texttt{x\_train}
already exists in the Python environment. Alternatively, one can also
define a TensorFlow placeholder,
\begin{lstlisting}
x = tf.placeholder(tf.float32, [N, D])
\end{lstlisting}
and feed the placeholder manually during inference.

A toy demonstration is available in the \href{/getting-started}{Getting Started} section.
Source code is available
\href{https://github.com/blei-lab/edward/blob/master/examples/bayesian_nn.py}
{here}.

\subsubsection{References}\label{references}
