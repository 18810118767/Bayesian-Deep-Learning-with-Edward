% Define the subtitle of the page
\title{KL(p||q) minimization}

% Begin the content of the page
\subsection{$\text{KL}(p\|q)$ minimization}

One form of variational inference minimizes the Kullback-Leibler divergence
\textbf{from} $p(z \mid x)$ \textbf{to} $q(z\;;\;\lambda)$,
\begin{align*}
  \lambda^*
  &=
  \arg\min_\lambda \text{KL}(
  p(z \mid x)
  \;\|\;
  q(z\;;\;\lambda)
  )\\
  &=
  \arg\min_\lambda\;
  \mathbb{E}_{p(z \mid x)}
  \big[
  \log p(z \mid x)
  -
  \log q(z\;;\;\lambda)
  \big].
\end{align*}
The KL divergence is a non-symmetric, information theoretic measure of
similarity between two probability distributions.

\subsubsection{Minimizing an intractable objective function}

The $\text{KL}(p\|q)$ objective we seek to minimize is intractable; it directly
involves the posterior $p(z \mid x)$. Ignoring this for the moment, consider its
gradient
\begin{align*}
  \nabla_\lambda\;
  \text{KL}(
  p(z \mid x)
  \;\|\;
  q(z\;;\;\lambda)
  )
  &=
  -
  \mathbb{E}_{p(z \mid x)}
  \big[
  \nabla_\lambda\;
  \log q(z\;;\;\lambda)
  \big].
\end{align*}
Both $\text{KL}(p\|q)$ and its gradient are intractable
because of the posterior expectation.
We can use importance sampling to both
estimate the objective and calculate stochastic gradients
\citep{oh1992adaptive}.

\subsubsection{Adaptive Importance sampling}

First rewrite the expectation to be with respect to the variational
distribution,
\begin{align*}
  -
  \mathbb{E}_{p(z \mid x)}
  \big[
  \nabla_\lambda\;
  \log q(z\;;\;\lambda)
  \big]
  &=
  -
  \mathbb{E}_{q(z\;;\;\lambda)}
  \Bigg[
  \frac{p(z \mid x)}{q(z\;;\;\lambda)}
  \nabla_\lambda\;
  \log q(z\;;\;\lambda)
  \Bigg].
\end{align*}

We then use importance sampling to obtain a noisy estimate of this gradient.
The basic procedure follows these steps:
\begin{enumerate}
  \item draw $S$ samples $\{z_s\}_1^S \sim q(z\;;\;\lambda)$,
  \item evaluate $\nabla_\lambda\; \log q(z_s\;;\;\lambda)$,
  \item compute the normalized importance weights
  \begin{align*}
    w_s
    &=
    \frac{p(z_s \mid x)}{q(z_s\;;\;\lambda)}
    \Bigg/
    \sum_{s=1}^{S}
    \frac{p(z_s \mid x)}{q(z_s\;;\;\lambda)}
  \end{align*}
  \item compute the weighted mean.
\end{enumerate}
The key insight is that we can use the joint $p(x,z)$ instead of the posterior
when estimating the normalized importance weights
\begin{align*}
  w_s
  &=
  \frac{p(z_s \mid x)}{q(z_s\;;\;\lambda)}
  \Bigg/
  \sum_{s=1}^{S}
  \frac{p(z_s \mid x)}{q(z_s\;;\;\lambda)} \\
  &=
  \frac{p(x, z_s)}{q(z_s\;;\;\lambda)}
  \Bigg/
  \sum_{s=1}^{S}
  \frac{p(x, z_s)}{q(z_s\;;\;\lambda)}.
\end{align*}
This follows from Bayes' rule
\begin{align*}
  p(z \mid x)
  &=
  p(x, z) / p(x)\\
  &=
  p(x, z) / \text{a constant function of }z.
\end{align*}

Importance sampling thus gives the following noisy yet unbiased gradient
estimate
\begin{align*}
\nabla_\lambda\;
  \text{KL}(
  p(z \mid x)
  \;\|\;
  q(z\;;\;\lambda)
  )
  &=
  -
  \frac{1}{S}
  \sum_{s=1}^S
  w_s
  \nabla_\lambda\; \log q(z_s\;;\;\lambda).
\end{align*}
The objective $\text{KL}(p\|q)$ can be calculated in a similar fashion.
The only new ingredient for its gradient is the score function
$\nabla_\lambda \log q(z\;;\;\lambda)$.  Edward uses automatic
differentiation, specifically with TensorFlow's computational graphs,
making this gradient computation both simple and efficient to
distribute.

Adaptive importance sampling follows this gradient to a local optimum using
stochastic optimization. It is adaptive because the variational distribution
$q(z\;;\;\lambda)$ iteratively gets closer to the posterior $p(z \mid x)$.

\subsubsection{Implementation}

We implement $\text{KL}(q\|p)$ minimization with the importance
sampling gradient in the class \texttt{KLpq}. It inherits from
\texttt{VariationalInference}, which provides a collection of default
methods such as an optimizer.

\begin{lstlisting}[language=Python]
class KLpq(VariationalInference):
  def __init__(self, *args, **kwargs):
    super(KLpq, self).__init__(*args, **kwargs)

  def initialize(self, n_samples=1, *args, **kwargs):
    self.n_samples = n_samples
    return super(KLpq, self).initialize(*args, **kwargs)

  def build_loss(self):
    x = self.data
    z = {key: rv.sample([self.n_samples])
         for key, rv in six.iteritems(self.latent_vars)}

    # normalized importance weights
    q_log_prob = 0.0
    for key, rv in six.iteritems(self.latent_vars):
      q_log_prob += tf.reduce_sum(rv.log_prob(tf.stop_gradient(z[key])),
                                  list(range(1, len(rv.get_batch_shape()) + 1)))

    log_w = self.model_wrapper.log_prob(x, z) - q_log_prob
    log_w_norm = log_w - log_sum_exp(log_w)
    w_norm = tf.exp(log_w_norm)

    self.loss = tf.reduce_mean(w_norm * log_w)
    return -tf.reduce_mean(q_log_prob * tf.stop_gradient(w_norm))
\end{lstlisting}

Two methods are added: \texttt{initialize()} and
\texttt{build_loss()}. The method \texttt{initialize()}
follows the same initialization as \texttt{VariationalInference}, and
adds an argument: \texttt{n_samples} for the number of samples from
the variational model.

The method \texttt{build_loss()} implements the $\text{KL}(q\|p)$
objective and its gradient. It draws \texttt{self.n_samples} samples from the
variational model. It registers the Monte Carlo
estimate of $\text{KL}(q\|p)$ in TensorFlow's computational graph, and stores it
in \texttt{self.loss}, to track progress of the inference for diagnostics.

The method returns an object whose
automatic differentiation is a stochastic gradient of $\text{KL}(p\|q)$.
The TensorFlow function \texttt{tf.stop_gradient()} tells the computational
graph to stop traversing nodes to propagate gradients. In this case,
the only gradients taken are $\nabla_\lambda \log q(z_s\;;\;\lambda)$,
one for each sample $z_s$. We multiply it by \texttt{w_norm}
element-wise and return the mean.

Computing the normalized importance weights is a numerically
challenging task.  Underflow is a common issue. Here we use the
\texttt{log_sum_exp} trick to compute weights in the log space,
available in \texttt{edward.util}. However, the normalized weights are
still exponentiated to compute the gradient. As with importance
sampling in general, this inference method does not scale to high
dimensions.

See the \href{/api/}{API} for more details.

\subsubsection{References}\label{references}
