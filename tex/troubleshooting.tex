\title{Troubleshooting}

\subsection{Troubleshooting}

\subsubsection{Basic Installation}

Edward depends on

\begin{itemize}
  \item NumPy (>=1.7)
  \item Six (>=1.1.0)
  \item TensorFlow (>=0.11.0rc0)
\end{itemize}

We recommend using \texttt{pip} to install \texttt{numpy} and
\texttt{six} as

\begin{lstlisting}[language=JSON]
pip install numpy six
\end{lstlisting}

\href{https://www.tensorflow.org/}{TensorFlow} is not on PyPI yet. We recommend
using \texttt{pip} to install the latest TensorFlow binaries. Please
follow the
\href
{https://www.tensorflow.org/versions/master/get_started/os_setup.html#pip-installation}
{instructions} from Google.

\subsubsection{TensorFlow via \texttt{conda}}

If you use the Anaconda Python distribution or the \texttt{conda}
Python package manager, getting started with the CPU-only version
of TensorFlow can be more simply
installed via the community-supported \texttt{conda-forge} channel:

\begin{lstlisting}[language=JSON]
conda config --add channels conda-forge
conda create -n cpu-tf-edward numpy six tensorflow
source activate cpu-tf-edward
pip install edward
\end{lstlisting}

The GPU-enabled version of TensorFlow can later be installed as
described above.

\subsubsection{Full Installation}

Edward has optional features that depend on external packages.

\begin{itemize}
  \item Neural networks are supported through three
  libraries:
  \href{http://keras.io}{Keras} (>=1.0)
\begin{lstlisting}[language=JSON]
pip install keras
\end{lstlisting}
\href{https://github.com/tensorflow/tensorflow/tree/master/tensorflow/contrib/slim}{TensorFlow
Slim} (native in TensorFlow), and
\href{https://github.com/google/prettytensor}{PrettyTensor} (>=0.5.3)
\begin{lstlisting}[language=JSON]
pip install prettytensor
\end{lstlisting}
  \item The \href{http://mc-stan.org}{Stan} modeling language is supported
  through PyStan (>=2.0.1.3)
\begin{lstlisting}[language=JSON]
pip install pystan
\end{lstlisting}
  \item The \href{http://pymc-devs.github.io/pymc3/}{PyMC3} modeling language is supported
  through PyMC3 (>=3.0). Please see their installation
  \href{http://pymc-devs.github.io/pymc3/notebooks/getting_started.html}
  {instructions}.
\end{itemize}
