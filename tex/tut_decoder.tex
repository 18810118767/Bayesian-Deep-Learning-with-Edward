% Define the subtitle of the page
\title{Probabilistic decoder}

% Begin the content of the page
\subsection{Probabilistic decoder}

A probabilistic decoder is a reinterpretation of model likelihoods
based on coding theory. It is a distribution $p(\mathbf{x}_n\mid \mathbf{z}_n)$  over each value
$\mathbf{x}_n\in\mathbb{R}^D$ given a code $\mathbf{z}_n$. The latent
variables $\mathbf{z}_n$ are interpreted as the hidden representation, or code, of the value
$\mathbf{x}_n$. The decoder is probabilistic because its generated
values (decoding) for any given code is random.

For real-valued data,
the randomness in the decoder is given by a multivariate Gaussian
\begin{align*}
  p(\mathbf{x}_n\mid\mathbf{z}_n)
  &=
  \mathcal{N}(\mathbf{x}_n\mid [\mu,\sigma^2]=\mathrm{NN}(\mathbf{z}_n; \mathbf{\theta})),
\end{align*}
where the probabilistic decoder is parameterized by a neural network
$\mathrm{NN}$ taking the code $\mathbf{z}_n$ as input.
For binary data, it is given by a Bernoulli
\begin{align*}
  p(\mathbf{x}_n\mid\mathbf{z}_n)
  &=
  \text{Bernoulli}(\mathbf{x}_n\mid p=\mathrm{NN}(\mathbf{z}_n; \mathbf{\theta})).
\end{align*}
Probabilistic decoders are typically used alongside a standard normal
prior over the code
\begin{align*}
  p(\mathbf{z})
  &=
  \mathcal{N}(\mathbf{z} \;;\; \mathbf{0}, I).
\end{align*}

Let's build the model in Edward using TensorFlow, and with
PrettyTensor as an easy way to build neural networks. Here we use a
probabilistic decoder to model binarized 28 x 28
pixel images from MNIST.
\begin{lstlisting}[language=Python]
class NormalBernoulli:
    """
    Each binarized pixel in an image is modeled by a Bernoulli
    likelihood. The success probability for each pixel is the output
    of a neural network that takes samples from a normal prior as
    input.

    p(x, z) = Bernoulli(x | p = neural_network(z)) Normal(z; 0, I)
    """
    def __init__(self, n_vars):
        self.n_vars = n_vars # number of local latent variables

    def neural_network(self, z):
        """p = neural_network(z)"""
        with pt.defaults_scope(activation_fn=tf.nn.elu,
                               batch_normalize=True,
                               learned_moments_update_rate=0.0003,
                               variance_epsilon=0.001,
                               scale_after_normalization=True):
            return (pt.wrap(z).
                    reshape([N_MINIBATCH, 1, 1, self.n_vars]).
                    deconv2d(3, 128, edges='VALID').
                    deconv2d(5, 64, edges='VALID').
                    deconv2d(5, 32, stride=2).
                    deconv2d(5, 1, stride=2, activation_fn=tf.nn.sigmoid).
                    flatten()).tensor

    def log_lik(self, xs, z):
        """
        Bernoulli log-likelihood, summing over every image n and pixel i
        in image n.

        log p(x | z) = log Bernoulli(x | p = neural_network(z))
         = sum_{n=1}^N sum_{i=1}^{28*28} log Bernoulli (x_{n,i} | p_{n,i})
        """
        return tf.reduce_sum(bernoulli.logpmf(xs['x'], p=self.neural_network(z)))

model = NormalBernoulli(n_vars=10)
\end{lstlisting}

An example script using this model can found
\href{https://github.com/blei-lab/edward/blob/master/examples/convolutional_vae.py}
{here}.
%We experiment with this model in the
%\href{variational_autoencoder.html}{variational auto-encoder} tutorial.

\subsubsection{References}\label{references}

\begin{itemize}
\item
  Dayan, P., Hinton, G. E., Neal, R. M., & Zemel, R. S. (1995). The Helmholtz Machine. Neural Computation, 7(5), 889–904.
\item
  Kingma, D. P., & Welling, M. (2014). Auto-Encoding Variational Bayes. In International Conference on Learning Representations.
\end{itemize}
