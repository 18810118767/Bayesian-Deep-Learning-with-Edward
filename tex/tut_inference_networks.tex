% Define the subtitle of the page
\title{Inference networks}

% Begin the content of the page
\subsection{Inference networks}

An inference network is a flexible construction for parameterizing
approximating distributions during inference.
They are used in Helmholtz machines (Dayan et al., 2000), deep
Boltzmann machines (Salakhutdinov and Larochelle, 2010), and
variational auto-encoders (Kingma & Welling, 2014; Rezende et al.,
2014).

Recall that probabilistic models often have \emph{local} latent
variables, that is, latent variables associated with a data point; for
example, the latent cluster assignment of a data point or the hidden
representation of a text or image.
Variational
inference on models with local latent variables
requires optimizing over each variational factor,
\begin{equation*}
q(\mathbf{z}; \mathbf{\lambda}) = \prod_{n=1}^N q(z_n; \lambda_n),
\end{equation*}
where $z_n$ are the latent variables associated to a data point $x_n$.
The set of parameters $\mathbf{\lambda}=\{\lambda_n\}$ grows with the
size of data. This makes computation difficult when the data
does not fit in memory.
Further, a new set of parameters $\mathbf{\lambda}'=\{\lambda_n'\}$
must be optimized at test time, where there may be a new set of data
points $\{x_n'\}$.

An inference network replaces local variational parameters with global
parameters. It is a neural network
which takes $x_n$ as input and which outputs its local variational parameters
$\lambda_n$,
\begin{equation*}
q(\mathbf{z}\mid \mathbf{x}; \theta)
= \prod_{n=1}^N q(z_n \mid x_n; \lambda_n)
= \prod_{n=1}^N q(z_n; \lambda_n = \mathrm{NN}(x_n; \theta)).
\end{equation*}
This amortizes inference by only defining a set of \emph{global} parameters,
namely, the parameters of the neural network. These parameters have a
fixed size, making the memory complexity of inference a constant.
Further, the same set of parameters can be used at test time:
regardless of seeing new or old data points $x_n'$, we simply pass it
through the neural network (a forward pass) to obtain its
corresponding variational factor $q(z_n'; \lambda_n' =
\mathrm{NN}(x_n'; \theta))$.

Note that the inference network is an approximation: it is a common
misunderstanding that the inference network produces a more expressive
variational model. The inference network restricts the size of
parameters, meaning it can only do as well as an approximation as the
original variational distribution without the inference network.

There are often good reasons beyond computational reasons
for wanting to use an inference network, depending on the problem
setting.
Originating from
Helmholtz machines (Dayan et al., 2000), they are motivated by the hypothesis
that "the human perceptual system is a statistical inference engine
whose function is to infer the probable causes of sensory input." The
inference network is this function.
In cognition, "humans and robots are in the setting of amortized
inference: they have to solve many similar inference problems, and can
thus offload part of the computational work to shared precomputation
and adaptation over time" (Stuhlmuller et al., 2013).

\subsubsection{Implementation}

Inference networks are easy to build in Edward.
In the example below, a data point \texttt{x} is a 28 by 28 pixel
image (from MNIST).
We model each \texttt{x} with 10 normally distributed latent
variables.
We build a convolutional neural network using PrettyTensor.
\begin{lstlisting}[language=Python]
N_VARS = 10 # number of local latent variables

def neural_network(x):
    with pt.defaults_scope(activation_fn=tf.nn.elu,
                           batch_normalize=True,
                           learned_moments_update_rate=0.0003,
                           variance_epsilon=0.001,
                           scale_after_normalization=True):
        params = (pt.wrap(x).
                reshape([28, 28, 1]).
                conv2d(5, 32, stride=2).
                conv2d(5, 64, stride=2).
                conv2d(5, 128, edges='VALID').
                dropout(0.9).
                flatten().
                fully_connected(N_VARS * 2, activation_fn=None)).tensor

    # Return list of vectors where mean[i], stddev[i] are the
    # parameters of the local variational factor for data point i.
    loc = tf.reshape(params[:, :N_VARS], [-1])
    scale = tf.reshape(tf.sqrt(tf.exp(params[:, N_VARS:])), [-1])
    return [loc, scale]
\end{lstlisting}
The neural network
takes a 28 by 28 pixel image and outputs a $2\cdot 10$-dimensional
output, one for the mean (\texttt{loc}) and one for the standard
deviation (\texttt{scale}).

We define a TensorFlow placeholder \texttt{x_ph} for the neural network's input.
We define a variational model which is parameterized by the neural network's
output.
\begin{lstlisting}[language=Python]
x_ph = tf.placeholder(tf.float32, [28 * 28])
loc, scale = neural_network(x_ph)
variational = Variational()
variational.add(Normal(N_VARS, loc=loc, scale=scale))
\end{lstlisting}
During each step of inference, we pass in a data point to feed the
placeholder. Then we train the variational parameters (inference
network's parameters) according to
gradients of the variational objective.

An example script using this variational model can found
\href{https://github.com/blei-lab/edward/blob/master/examples/convolutional_vae.py}
{here}.
%We experiment with this model in the
%\href{variational_autoencoder.html}{variational auto-encoder} tutorial.

\subsubsection{Footnotes}

Inference networks are also known as recognition models, recognition
networks, or inverse mappings.

This tutorial is taken from text originating in Tran et al. (2016).
We thank Kevin Murphy for motivating the tutorial as it is based
on our discussions, and also related discussion with Jaan Altosaar.

\subsubsection{References}\label{references}

\begin{itemize}
\item
  Dayan, P., Hinton, G. E., Neal, R. M., & Zemel, R. S. (1995). The
  Helmholtz Machine. Neural Computation, 7(5), 889–904.
\item
  Kingma, D. P., & Welling, M. (2014). Auto-Encoding Variational
  Bayes. In International Conference on Learning Representations.
\item
  Rezende, D. J., Mohamed, S., & Wierstra, D. (2014). Stochastic
  Backpropagation and Approximate Inference in Deep Generative Models.
  In International Conference on Machine Learning.
\item
  Salakhutdinov, R., \& and Larochelle, H. (2010). Efficient learning of
  deep Boltzmann machines. In Artificial Intelligence and Statistics.
\item
  Stuhlmüller, A., Taylor, J., & Goodman, N. (2013). Learning
  Stochastic Inverses. In Neural Information Processing Systems.
\item
  Tran, D., Ranganath, R., & Blei, D. M. (2016). The Variational
  Gaussian Process. In International Conference on Learning
  Representations.
\end{itemize}
