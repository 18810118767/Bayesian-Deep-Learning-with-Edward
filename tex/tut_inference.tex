% Define the subtitle of the page
\title{Inference of probability models}

% Begin the content of the page
\subsection{Inference of probability models}

This tutorial asks the question: what does it mean to do inference of
probability models? This sets the stage for understanding how to
design inference algorithms in Edward.

\subsubsection{The posterior}

How can we use a model $p(x,z)$ to analyze some data $x$? In other words,
what hidden structure $z$ explains the data? We seek to infer this
hidden structure using the model.

Inference of probability models leverages Bayes' rule to define the
\emph{posterior}
\begin{align*}
  p(z \mid x)
  &=
  \frac{p(x,z)}{\int p(x,z) \text{d}z}.
\end{align*}
The posterior is the distribution of the latent variables $z$, conditioned on
some (observed) data $x$.
Drawing analogy to representation learning, it is a probabilistic
description of the data's hidden representation.

From the perspective of inductivism, as practiced by classical
Bayesians (and implicitly by frequentists),
the posterior is our updated hypothesis about the latent variables.
From the perspective of hypothetico-deductivism, as practiced by
statisticians such as Box, Rubin, and Gelman, the posterior is simply
a fitted model to data, to be criticized and thus revised (Box, 1983;
Gelman and Shalizi, 2012).



\subsubsection{Inferring the posterior}

Now we know what the posterior represents. How do we calculate it? This is the
central computational challenge in inference.

The posterior is difficult to compute because of its normalizing
constant, which is the integral in the denominator.
This is often a high-dimensional integral that lacks an analytic (closed-form)
solution. Thus, calculating the posterior means \emph{approximating} the
posterior.

There are many approaches to posterior inference, which can be unwieldy to
manage or even conceptualize. Edward uses classes and class
inheritance to provide a hierarchy of inference methods, enabling fast
experimentation on top of existing methods. For details on the inference
hierarchy in Edward, see the
\href{api/inferences.html}{inference API}. We describe several examples in
detail in the other inference \href{tutorials.html}{tutorials}.


\subsubsection{References}\label{references}

\begin{itemize}
\item
  Box, G. E. (1982). An Apology for Ecumenism in Statistics (No.
  MRC-TSR-2408). Wisconsin Univ-Madison Mathematics Research Center.
\item
  Gelman, A., & Shalizi, C. R. (2012). Philosophy and the practice of
  Bayesian statistics. British Journal of Mathematical and Statistical
  Psychology, 66(1), 8–38.
\end{itemize}
