\title{Latent space models for neural data}

\subsection{Latent space models for neural data}

Many interesting data sets exhibit a network structure. For example, social
networks or biological neural networks.

What we can learn about the nodes of the network from its connectivity patterns?
We can begin to study this using a latent space model.
Latent space models embed the nodes of the network in a latent space which
captures attributes of the nodes. The likelihood to form an edge between two
nodes depends on their distance in the latent space.

Let us study an dataset from from neuroscience.
The script can be found
\href{https://github.com/blei-lab/edward/blob/master/examples/latent_space_model.py}{here}.

\subsubsection{Data}

The data comes from \href{http://www-personal.umich.edu/~mejn/netdata/}{Mark Newman's repository}.
It is a weighted, directed network representing the neural network of
the nematode
\href{https://en.wikipedia.org/wiki/Caenorhabditis_elegans}{C.~Elegans}
compiled by Watts and Strogatz from original experimental data
by White and co-authors.

The neural network consists of around $300$ neurons. Each connection
is associated with a weight (positive integer) capturing the strength
of the connection.

In the example we load the data with
\begin{lstlisting}[language=Python]
def load_celegans_brain():
  x = np.load('data/celegans_brain.npy')
  N = x.shape[0]
  return {'x': x}, N

data, N = load_celegans_brain()
\end{lstlisting}

\subsubsection{Model}

Here we will fit a latent space model to the C.~Elegans neural network. What can
we learn about the neurons from how they are connected? We will learn a latent
embedding for each neuron to capture the similarities between them.

Each neuron $n$ is a node in the network and is associated with a latent
$K$-dimensional vector $z_n$.

We place a Gaussian prior on each of the latent vectors.
The log-odds of an edge between node $i$ and
$j$ is proportional to the Euclidean distance between the latent
representations of the nodes $|z_i- z_j|$. Here, we
model the weights ($Y_{ij}$) of the edges with a Poisson likelihood.
The rate is the reciprocal of the distance in latent space. This is
the generative process:

For each node $n$:
\begin{align}
z_n \sim N(0,I).
\end{align}
For each edge $(i,j)$:
\begin{align}
Y_{ij} \sim \text{Poisson}\Bigg(\frac{1}{|z_i - z_j|}\Bigg).
\end{align}

Here we build the model in Edward using TensorFlow.
\begin{lstlisting}[language=Python]
class LatentSpaceModel:
  """
  p(x, z) = [ prod_{i=1}^N prod_{j=1}^N Poi(Y_{ij}; 1/||z_i - z_j|| ) ]
        [ prod_{i=1}^N N(z_i; 0, I)) ]

  """
  def __init__(self, N, K, var=1.0,
               like='Poisson',
               prior='Lognormal',
               dist='euclidean'):
    self.n_vars = N * K
    self.N = N
    self.K = K
    self.prior_variance = var
    self.like = like
    self.prior = prior
    self.dist = dist

  def log_prob(self, xs, zs):
    """Return a vector [log p(xs, zs[1,:]), ..., log p(xs, zs[S,:])]."""
    zs = zs['z']
    if self.prior == 'Lognormal':
      zs = tf.exp(zs)
    elif self.prior != 'Gaussian':
      raise NotImplementedError("prior not available.")

    log_prior = -self.prior_variance * tf.reduce_sum(zs * zs)

    z = tf.reshape(zs, [self.N, self.K])
    if self.dist == 'euclidean':
      xp = tf.matmul(tf.ones([1, self.N]),
                     tf.reduce_sum(z * z, 1, keep_dims=True))
      xp = xp + tf.transpose(xp) - 2 * tf.matmul(z, z, transpose_b=True)
      xp = 1.0 / xp
    elif self.dist == 'cosine':
      xp = tf.matmul(z, z, transpose_b=True)

    if self.like == 'Gaussian':
      log_lik = tf.reduce_sum(norm.logpdf(xs['x'], xp))
    elif self.like == 'Poisson':
      if not (self.dist == 'euclidean' or self.prior == "Lognormal"):
        raise NotImplementedError("Rate of Poisson has to be nonnegatve.")

      log_lik = tf.reduce_sum(poisson.logpmf(xs['x'], xp))
    else:
      raise NotImplementedError("likelihood not available.")

    return log_lik + log_prior


K = 3
model = LatentSpaceModel(N, K=3, like='Poisson', prior='Gaussian')
\end{lstlisting}

\subsubsection{Inference}

Maximum a posteriori (MAP) estimation is simple in Edward. Two lines are
required: Instantiating inference and running it.
\begin{lstlisting}[language=Python]
inference = ed.MAP(['z'], data, model)
\end{lstlisting}

See this extended tutorial about
\href{tut_MAP}{MAP estimation in Edward}.

One could instead run variational inference. This requires specifying
a variational model and instantiating \texttt{MFVI}.
\begin{lstlisting}[language=Python]
qz = Normal(model.n_vars)
inference = ed.MFVI({'z': qz}, data, model)
\end{lstlisting}
Finally, the following line runs the inference procedure for 5000
iterations and prints progress every 500 iteration.
\begin{lstlisting}[language=Python]
inference.run(n_iter=5000, n_print=500)
\end{lstlisting}
See this extended tutorial about
\href{tut_variational_inference}{variational inference in Edward}.

\subsubsection{References}

\begin{itemize}
\item
White, J.~G., Southgate, E., Thomson, J.~N., and Brenner, S. (1986).
\newblock The structure of the nervous system of the nematode caenorhabditis
  elegans.
\newblock \emph{Philos Trans R Soc Lond B Biol Sci}, 314(1165):1--340.
\item
Watts, D.~J. and Strogatz, S.~H. (1998).
\newblock Collective dynamics of ‘small-world’ networks.
\newblock \emph{Nature}, 393(6684):440--442.
\item
Hoff, P.~D., Raftery, A.~E., and Handcock, M.~S. (2002).
\newblock Latent space approaches to social network analysis.
\newblock \emph{Journal of the american Statistical association},
  97(460):1090--1098.
\end{itemize}
