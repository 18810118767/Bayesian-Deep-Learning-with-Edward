% Define the subtitle of the page
\title{Variational inference}

% Begin the content of the page
\subsection{Variational inference}

Variational inference is an umbrella term for algorithms which cast
posterior inference as optimization. The core idea involves two steps:
\begin{enumerate}
   \item posit a family of distributions $q(z\;;\;\lambda)$ over the
   latent variables;
   \item match $q(z\;;\;\lambda)$ to the posterior by optimizing over its
   parameters $\lambda$.
 \end{enumerate}
This strategy converts the problem of computing the posterior $p(z \mid x)$ into
an optimization problem, which minimizes a divergence measure
\begin{align*}
  \lambda^*
  &=
  \arg\min_\lambda \text{divergence}(
  p(z \mid x)
  ,
  q(z\;;\;\lambda)
  ).
\end{align*}
The optimized distribution $q(z\;;\;\lambda)$ is then used as a
proxy to the posterior $p(z\mid x)$.

Edward takes the perspective that the posterior is (typically)
intractable, and thus we must build a model of latent variables that
best approximates the posterior.
It is analogous to the perspective
that the true data generating process is unknown, and thus we build
models of data to best approximate the true process.

For details on the variational inference base class defined in Edward,
or the design of variational models, see the
\href{api/inferences}{inference API}.
For examples of specific variational inference algorithms in
Edward, see the other inference \href{tutorials}{tutorials}.

\subsubsection{References}\label{references}

\begin{itemize}
\item
  Hinton, G. and Van Camp, D (1993). Keeping the neural networks
  simple by minimizing the description length of the weights. In
  Computational Learning Theory, pp. 5–13. ACM.
\item
  Jordan, M. I., Ghahramani, Z., Jaakkola, T. S., & Saul, L. K.
  (1999). An introduction to variational methods for graphical models.
  Machine Learning, 37(2), 183-233.
\item
  Waterhouse, S., MacKay, D., and Robinson, T (1996). Bayesian methods
  for mixtures of experts. In Neural Information Processing Systems.
\end{itemize}
