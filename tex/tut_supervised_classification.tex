% Define the subtitle of the page
\title{Supervised learning (Classification)}

% Begin the content of the page
\subsection{Supervised learning (Classification)}

In supervised learning, the task is to infer hidden structure from
labeled data, comprised of training examples $\{(x_n, y_n)\}$.
Classification means the output $y$ takes discrete values.

We demonstrate how to do this in Edward with an example.
The script is available
\href{https://github.com/blei-lab/edward/blob/master/examples/gp_classification.py}
{here}.


\subsubsection{Data}

Use 25 data points from the \href
{https://stat.ethz.ch/R-manual/R-devel/library/MASS/html/crabs.html}
{crabs data set}.
\begin{lstlisting}[language=Python]
df = np.loadtxt('data/crabs_train.txt', dtype='float32', delimiter=',')[:25, :]
data = {'x': df[:, 1:], 'y': df[:, 0]}
\end{lstlisting}


\subsubsection{Model}

Posit the model as Gaussian process classification. For more details on the
model, see the
\href{tut_gp_classification}
{Gaussian process classification tutorial}.

Here we build the model in Edward using TensorFlow.
\begin{lstlisting}[language=Python]
class GaussianProcess:
    """
    Gaussian process classification

    p((x,y), z) = Bernoulli(y | logit^{-1}(x*z)) *
                  Normal(z | 0, K),

    where z are weights drawn from a GP with covariance given by k(x,
    x') for each pair of inputs (x, x'), and with squared-exponential
    kernel and known kernel hyperparameters.

    Parameters
    ----------
    N : int
        Number of data points.
    sigma : float, optional
        Signal variance parameter.
    l : float, optional
        Length scale parameter.
    """
    def __init__(self, N, sigma=1.0, l=1.0):
        self.N = N
        self.sigma = sigma
        self.l = l

        self.n_vars = N
        self.inverse_link = tf.sigmoid

    def kernel(self, x):
        mat = []
        for i in range(self.N):
            mat += [[]]
            xi = x[i, :]
            for j in range(self.N):
                if j == i:
                    mat[i] += [multivariate_rbf(xi, xi, self.sigma, self.l)]
                else:
                    xj = x[j, :]
                    mat[i] += [multivariate_rbf(xi, xj, self.sigma, self.l)]

            mat[i] = tf.pack(mat[i])

        return tf.pack(mat)

    def log_prob(self, xs, zs):
        """Return a vector [log p(xs, zs[1,:]), ..., log p(xs, zs[S,:])]."""
        x, y = xs['x'], xs['y']
        log_prior = multivariate_normal.logpdf(zs, cov=self.kernel(x))
        log_lik = tf.pack([tf.reduce_sum(
            bernoulli.logpmf(y, self.inverse_link(tf.mul(y, z)))
            ) for z in tf.unpack(zs)])
        return log_prior + log_lik

model = GaussianProcess(N=len(df))
\end{lstlisting}


\subsubsection{Inference}

Perform variational inference.
Define the variational model to be a fully factorized normal
\begin{lstlisting}[language=Python]
variational = Variational()
variational.add(Normal(model.n_vars))
\end{lstlisting}

Run mean-field variational inference for 500 iterations.
\begin{lstlisting}[language=Python]
inference = ed.MFVI(model, variational, data)
inference.run(n_iter=500)
\end{lstlisting}
In this case
\texttt{MFVI} defaults to minimizing the
$\text{KL}(q\|p)$ divergence measure using the reparameterization
gradient.
For more details on inference, see the \href{tut_KLqp}{$\text{KL}(q\|p)$ tutorial}.
(This example happens to be slow because evaluating and inverting full
covariances in Gaussian processes happens to be slow.)


%\subsubsection{Criticism}
