% Define the subtitle of the page
\title{Bayesian Linear Regression}

% Begin the content of the page
\subsection{Bayesian Linear Regression}


Bayesian linear regression is a model $p(\mathbf{y}\mid \mathbf{X})$ of
outputs $y\in\mathbb{R}$, also known as the response, given
a vector of inputs
$\mathbf{x}\in\mathbb{R}^D$, also known as the features or covariates.
The model assumes a
linear relationship between these two random variables.

For a set of $N$ data points $(\mathbf{X},\mathbf{y})=\{(\mathbf{x}_n, y_n)\}$,
the model is
\begin{align*}
  p(\mathbf{z})
  &=
  \mathcal{N}(\mathbf{z} \;;\; \mathbf{0}, \sigma_z^2\mathbf{I}),
  \\
  p(\mathbf{y} \mid \mathbf{z}, \mathbf{X})
  &=
  \prod_{n=1}^N
  \mathcal{N}(y_n \;;\; \mathbf{x}_n^\top\mathbf{w} + b, \sigma_y^2).
\end{align*}
The latent variables $\mathbf{z}=[\mathbf{w},b]$ form the weights $\mathbf{w}$
and intercept $b$, also known as the bias.
Assume $\sigma_z^2$ is a known prior variance and $\sigma_y^2$ is a
known likelihood variance. The mean of the likelihood is given by a
linear transformation of the inputs $\mathbf{x}_n$ and shared weights
$\mathbf{w}$, along with the intercept $b$.

Let's build the model in Edward using TensorFlow, fixing $D=10$.
\begin{lstlisting}[language=Python]
class BayesianLinearRegression:
    """
    Bayesian linear regression for outputs y on inputs x.

    p((x,y), z) = Normal(y | x*z, lik_variance) *
                  Normal(z | 0, prior_variance),

    where z are weights, and with known lik_variance and
    prior_variance.

    Parameters
    ----------
    lik_variance : float, optional
        Variance of the normal likelihood; aka noise parameter,
        homoscedastic variance, scale parameter.
    prior_variance : float, optional
        Variance of the normal prior on weights; aka L2
        regularization parameter, ridge penalty, scale parameter.
    """
    def __init__(self, lik_variance=0.01, prior_variance=0.01):
        self.lik_variance = lik_variance
        self.prior_variance = prior_variance
        self.n_vars = 11

    def log_prob(self, xs, zs):
        """Return a vector [log p(xs, zs[1,:]), ..., log p(xs, zs[S,:])]."""
        x, y = xs['x'], xs['y']
        log_prior = -tf.reduce_sum(zs*zs, 1) / self.prior_variance
        # broadcasting to do (x*W) + b (n_minibatch x n_samples - n_samples)
        b = zs[:, 0]
        W = tf.transpose(zs[:, 1:])
        mus = tf.matmul(x, W) + b
        # broadcasting to do mus - y (n_minibatch x n_samples - n_minibatch x 1)
        y = tf.expand_dims(y, 1)
        log_lik = -tf.reduce_sum(tf.pow(mus - y, 2), 0) / self.lik_variance
        return log_lik + log_prior

model = BayesianLinearRegression()
\end{lstlisting}

We experiment with this model in the \href{tut_supervised_regression}{supervised
learning (regression)} tutorial.
An example script with visualization is available
\href{https://github.com/blei-lab/edward/blob/master/examples/bayesian_linear_regression_plot.py}
{here}.

\subsubsection{References}\label{references}

\begin{itemize}
\item
  Murphy, K. (2012). Machine Learning: A Probabilistic Perspective. MIT Press.
\end{itemize}
